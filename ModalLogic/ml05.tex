\documentclass{article}
\title{{\sc Philosophy Core Seminar Essay}\\ {\it \paperauthor~-- \papertitle}}
\author{Egbert Rijke}
\date\today

\usepackage{mathpazo}
\usepackage{amsmath,amssymb,amsthm}


\newcommand{\hwnumber}{5}

\begin{document}
\maketitle

\begin{ex}
\end{ex}

\begin{ex}
\end{ex}

\begin{ex}
Recall that $\varphi\to\psi$ is valid if
and only if $\nu(\varphi)\subseteq\nu(\psi)$ for all valuations $\nu$. 
\begin{enumerate}
\item The statement $\nec(\varphi\land\psi)\leftrightarrow(\nec\varphi\land\nec\psi)$
is valid because
\begin{align*}
\nu(\nec(\varphi\land\psi))
  & =
\interior{\nu(\varphi\land\psi)}
  \\
  & =
\interior{\nu(\varphi)\cap\nu(\psi)}
  \\
  & =
\interior{\nu(\varphi)}\cap\interior{\nu(\psi)}
  \\
  & =
\nu(\nec\varphi)\cap\nu{\nec\psi}
  \\
  & =
\nu(\nec\varphi\land\nec\psi)
\end{align*}
for any valuation $\nu$.
\item Suppose that $\varphi\to\psi$ is valid. Then $\nu(\varphi)\subseteq\nu(\psi)$
for any valuation $\nu$. The interior operation is monotone, so it follows that
$\interior{\nu(\varphi)}\subseteq\interior{\nu(\psi)}$. Thus we conclude that
$\nu(\nec\varphi)\subseteq\nu(\nec\psi)$ for every valuation $\nu$, 
and hence that $\nec\varphi\to\nec\psi$ is valid.
\end{enumerate}
\end{ex}

\begin{ex}
\begin{enumerate}
\item Suppose $\varphi\to\nec\varphi$ is valid in a space $X$. Then, for every
$A\subseteq X$ we have $A\subseteq\interior{A}$. Thus, it follows that
$A=\interior{A}$ for every $A\subseteq X$. In other words, every subset of $X$
is open. We conclude that $X$ is discrete.

Now suppose that $X$ is discrete, let $\varphi$ be a modal formula and let
$\nu$ be a valuation. Then $\nu(\varphi)$ is open, so we have
\begin{equation*}
\nu(\varphi)=\interior{\nu(\varphi)}=\nu(\nec\varphi).
\end{equation*}
It follows that $\nu(\varphi\to\nec\varphi)=X$, i.e.~that $\varphi\to\nec\varphi$
is valid in $X$.
\item To figure out a candidate formula, let $p$ be an atomic proposition and let 
$X$ be a space. Then the function
$\{\nu(p)\mid\nu\text{ is a valuation}\}\to 2^X$ is surjective. Thus, we see that
every closed subset of $X$ can be obtained as a valuation of $\pos p$. Asserting
that the valuation of $\pos p$ should always be open can be done by asserting that
$\pos p\to \nec\pos p$. We will show that $\pos\varphi\to\nec\pos\varphi$ is
defines the class of spaces of which closed sets are open. By the previous reasoning,
we have already seen that if $\pos\varphi\to\nec\pos\varphi$ is valid in $X$, then
every closed set is open.

For the other direction, let $X$ be a space in which all closed sets are open, 
let $\varphi$ be a modal formula and let $\nu$ be a valuation. Then
$\nu(\pos\varphi)=\closure{\nu(\varphi)}$ is open by assumption, so
$\closure{\nu(\varphi)}\subseteq\interior{\closure{\nu(\varphi)}}
=\nu(\nec\pos\varphi)$. It follows that $\nu(\pos\varphi\to\nec\pos\varphi)=X$,
and hence that $\pos\varphi\to\nec\pos\varphi$ is valid in $X$. 
\end{enumerate}
\end{ex}

\begin{ex}
\begin{enumerate}
\item Define $U_x:=\bigcap\{U\in\tau\mid x\in U\}$. Then $U_x$ is the intersection
of open sets, hence it is open. It is contained in every open set which contains
$x$. To see that $x\in U_x$, note that $X\in\{U\in\tau\mid x\in U\}$.
\item Suppose that every point in $X$ has a minimal neighborhood $U_x$ and let
$V:I\to\tau$ for some set $I$. The goal is to show that $V:=\bigcap_{i\in I}V(i)$
is open. To do this, it suffices to show that $U_x\subseteq V$ for every
$x\in V$. Suppose $x\in V$. Then $x\in V(i)$ for each $i\in I$. Since $U_x$ is
minimal, we also have $U_x\subseteq V(i)$ for each $i\in I$. Thus we conclude
that $U_x\subseteq V$.
\end{enumerate}
\end{ex}
\end{document}
