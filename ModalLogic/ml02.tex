\documentclass{article}
\title{Formal Methods homework set \hwnumber}
\author{Egbert Rijke}
\date\today

\usepackage{mathpazo}
\usepackage{amsmath,amsthm,amssymb}
\usepackage{xcolor}
\usepackage{comment}

\newcommand{\note}[1]{{\color{red}#1}}

\newcommand{\sigalg}{\mathcal}

\theoremstyle{definition}
\newtheorem{ex}{Exercise}



\newcommand{\hwnumber}{2}

\begin{document}
\maketitle

\begin{ex}
\begin{enumerate}
\item The trivial model $M$ with one point $x$ satisfies $(M,x)\models\nec\bot$
while $(M,x)\nmodels\pos\bot$. So this is a countermodel. By soundness, there
can't be a $K$-proof of $\nec\bot\to\pos\bot$. 
\item Now consider the frame $\mathbb{N}$ with the binary relation $S$ satisfying
$n S m$ if and only if $n+1=m$. Since every natural number has a successor, 
is immediate that this frame validates $\nec\varphi\to\pos\varphi$. Now consider 
a valuation with $\lsem p\rsem:=\{1\}$. Then we have $(M,0)\models\nec p$ while
$(M,0)\nmodels p$. 
\item Recall that $\pos$ is defined to be $\neg\nec\neg$. The following is a 
$\mathsf{KT}$ proof of the formula $\nec\varphi\to\neg\nec\neg\varphi$:
\begin{align*}
& \nec\neg\varphi\to\neg\varphi
  \tag{axiom}
  \\
& (\nec\neg\varphi\to\neg\varphi)\to(\neg\neg\varphi\to\neg\nec\neg\varphi)
  \tag{axiom}
  \\
& \neg\neg\varphi\to\neg\nec\neg\varphi
  \tag{modus ponens}
  \\
& (\neg\neg\varphi\to\neg\nec\neg\varphi)\to(\varphi\to\neg\nec\neg\varphi)
  \tag{axiom}
  \\
& \varphi\to\neg\nec\neg\varphi
  \tag{modus ponens}
  \\
& \nec\varphi\to\varphi
  \tag{axiom}
  \\
& (\nec\varphi\to\varphi)\to(\varphi\to\neg\nec\neg\varphi)\to(\nec\varphi\to\neg\nec\neg\varphi)
  \tag{axiom}
  \\
& (\varphi\to\neg\nec\neg\varphi)\to(\nec\varphi\to\neg\nec\neg\varphi)
  \tag{modus ponens}
  \\
& \nec\varphi\to\neg\nec\neg\varphi
  \tag{modus ponens}
\end{align*}
\end{enumerate}
\end{ex}

\begin{ex}
\begin{enumerate}
\item Note that the frames validating $\nec\varphi\to\varphi$ are the reflexive frames and the
frames validating $\nec\varphi\to\nec\nec\varphi$ are the transitive frames. Thus, $\mathsf{S4}$
defines the class of reflexive, transitive frames. 

Now consider the frame $\{0,1\}$ with the natural order $\leq$. By the above, this validates $\mathsf{S4}$. Let $\lsem p\rsem:=\{0\}$. Then we have $(M,0)\models p$. On the other hand, we have $(M,1)\nmodels\pos p$, so we also have $(M,0)\nmodels\nec\pos p$. Therefore, $(M,0)\nmodels p\to\nec\pos p$. By soundness, it follows
that there is no $\mathsf{S4}$ proof of $p\to\nec\pos p$.
\item In words, the formula says that if there exists an accessible state $y$ with the property that all states accessible from $y$ satisfy $\varphi$, then every accessible state satisfies $\varphi$. 
\begin{align*}
& \neg\varphi\to\nec\pos\neg\varphi
  \tag{axiom}
  \\
& (\neg\varphi\to\nec\pos\neg\varphi)\to(\neg\nec\pos\neg\varphi\to\neg\neg\varphi)
  \tag{axiom}
  \\
& \neg\nec\pos\neg\varphi\to\neg\neg\varphi
  \tag{modus ponens}
  \\
& (\neg\nec\pos\neg\varphi\to\neg\neg\varphi)\to(\neg\nec\neg\neg\pos\neg\varphi\to\varphi)
  & &
  \tag{axiom}
  \\
& \neg\nec\neg\neg\pos\neg\varphi\to\varphi
  \tag{axiom}
  \\
& \pos\neg\neg\nec\neg\neg\varphi\to\varphi
  \tag{definition}
  \\
& (\pos\neg\neg\nec\neg\neg\varphi\to\varphi)\to(\pos\nec\varphi\to\varphi)
  \tag{axiom}
  \\
& \pos\nec\varphi\to\varphi
  \tag{modus ponens}
  \\
& \pos\nec\varphi\to\nec\varphi
  \tag{goal to reach}
\end{align*} 
\end{enumerate}
\end{ex}

\begin{ex}
The proof is by induction.
\begin{itemize}
\item The formula $\nec\neg\bot$ is valid for all frames since $\neg\bot$ is valid for all frames.
\item To show that $(M,x)\models\nec(\varphi\land\psi)\leftrightarrow(\nec\varphi\land\nec\psi)$
note that 
\begin{itemize}
\item $(M,x)\models\nec(\varphi\land\psi)$; if and only if
\item $(M,y)\models\varphi\land\psi$ for every $y$ accessible from $x$; if and only if
\item both $(M,y)\models\varphi$ for every $y$ accessible from $x$, and $(M,y)\models\psi$ for every
$y$ accessible from $x$; if and only if
\item both $(M,x)\models\nec\varphi$ and $(M,x)\models\nec\psi$; if and only if
\item $(M,x)\models\nec\varphi\land\nec\psi$.
\end{itemize}
\item We have seen in class that the formula $\nec\varphi\to\varphi$ defines the reflexive
frames.
\item We have seen in class that the formula $\nec\varphi\to\nec\nec\varphi$ defines the the
transitive frames.
\item We have seen in class that every frame preserves validity via modus ponens.
\item It remains to show that if every reflexive, transitive frame validates $\varphi\to\psi$,
then every reflexive, transitive frame also validates $\nec\varphi\to\nec\psi$. Thus, assume $(M,x)\nmodels\varphi\to\psi$
for every world $x$ in a model $M$ over a reflexive, transitive frame. Let $M$ be a model
based on a reflexive transitive frame and let $x$ be a world in it. We have to verify that
$(M,x)\models\nec\varphi\to\nec\psi$. Suppose therefore that $(M,x)\models\nec\varphi$. Then
$(M,y)\models\varphi$ for every $y$ accessible from $x$. Let $y$ be accessible from $x$.
By assumption we have that
$(M,y)\models\varphi\to\psi$. Therefore it follows that $(M,y)\models\psi$. So we conclude
that $(M,y)\models\psi$ for every world $y$ accessible from $x$. Therefore we also have
that $(M,x)\models\nec\psi$. Hence $(M,x)\models\nec\varphi\to\nec\psi$.
\end{itemize} 
\end{ex}

\begin{ex}
\begin{enumerate}
\item Consider the frames $M$ with two points and $N$ with one of the points of $M$, both with the least reflexive relation. Then $N$ is a generated subframe of $M$. Thus the point of $N$ has the same theory with respect to $N$ as with respect to $M$. However, both points of $M$ satisfy $\neg\bot$. This shows that $R\neg\bot$ does not hold in $M$, but it does in $N$. Therefore, $R$ is not equivalent to any formula of the basic modal language.
\item Consider the primitive proposition $p$ with 
\begin{equation*}
\lsem p\rsem:=\bigcup_{k\in\mathbb{Z}}[2k-1,2k).
\end{equation*} 
For this primitive proposition, we see that
$(\mathbb{R},0)\models Sp$ while $(\mathbb{R},\frac{1}{2})\nmodels Sp$.

Now we define the bisimilation
$\sim$ by
\begin{align*}
x & \sim y & & \Leftrightarrow & & x\models p\text{ if and only if }y\models p
\end{align*}
We verify the three conditions that this is indeed a bisimilation. The first condition is satisfied by construction since $p$ is the only primitive proposition under consideration.
\begin{itemize}
\item Suppose that $x\sim y$ and $x<x'$. 
\begin{itemize}
\item If $x'\models p$ and $y\models p$ then $x'\sim y+2$; if $x'\models p$ and $y\nmodels p$ then $x'\sim y+1$.
\item If $x'\nmodels p$ and $y\models p$ then $x'\sim y+1$; if $x'\nmodels p$ and $y\nmodels p$ then $x'\sim y+2$.
\end{itemize}
\item Suppose that $x\sim y$ and $x'<x$.
\begin{itemize}
\item If $x'\models p$ and $y\models p$ then $x'\sim y-2$; if $x'\models p$ and $y\nmodels p$ then $x'\sim y-1$.
\item If $x'\nmodels p$ and $y\models p$ then $x'\sim y-1$; if $x'\nmodels p$ and $y\nmodels p$ then $x'\sim y-2$.
\end{itemize}
\item Suppose that $x\sim y$ and $y<y'$.
\begin{itemize}
\item If $y'\models p$ and $x\models p$ then $x+2\sim y'$; if $y'\models p$ and $x\nmodels p$ then $x+1\sim y'$.
\item If $y'\nmodels p$ and $x\models p$ then $x+1\sim y'$; if $y'\nmodels p$ and $x\nmodels p$ then $x+2\sim y'$.
\end{itemize}
\item Suppose that $x\sim y$ and $y'<y$.
\begin{itemize}
\item If $y'\models p$ and $x\models p$ then $x-2\sim y'$; if $y'\models p$ and $x\nmodels p$ then $x-1\sim y'$.
\item If $y'\nmodels p$ and $x\models p$ then $x-1\sim y'$; if $y'\nmodels p$ and $x\nmodels p$ then $x-2\sim y'$.
\end{itemize}
\end{itemize}
This shows that $\sim$ is a bisimilation of bidirectional frames. Therefore $0$ and $\frac{1}{2}$ must have the same theories. Since $0\models Sp$ and $\frac{1}{2}\nmodels Sp$ while $0\sim\frac{1}{2}$ it follows that $S$ cannot be equivalent to a formula in the temporal language.
\end{enumerate}
\end{ex}

\begin{ex}
As in exercise 4(b), the bisimilation $\sim$ has the property that $x\sim y$ precisely when $x\models p$ if and only if $y\models p$. So the condition that bisimilar elements make the same primitive propositions true is trivially satisfied. The rest of the proof also goes very much like the proof of 4(b). For $k\in\mathbb{Z}$, write $\pi_3(k)\in\{0,1,2\}$ for the remainder of $k$ modulo $3$.
\begin{itemize}
\item Suppose that $x\sim y$ and $x<x'$. 
\begin{itemize}
\item If $x'\models p$ and $y\models p$ then $x'\sim y+3$; if $x'\models p$ and $y\nmodels p$ then $x'\sim y+3-\pi_3(y)$.
\item If $x'\nmodels p$ and $y\models p$ then $x'\sim y+1$; if $x'\nmodels p$ and $y\nmodels p$ then $x'\sim y+3$.
\end{itemize}
\item Suppose that $x\sim y$ and $x'<x$.
\begin{itemize}
\item If $x'\models p$ and $y\models p$ then $x'\sim y-3$; if $x'\models p$ and $y\nmodels p$ then $x'\sim y-\pi_3(y)$.
\item If $x'\nmodels p$ and $y\models p$ then $x'\sim y-1$; if $x'\nmodels p$ and $y\nmodels p$ then $x'\sim y-3$.
\end{itemize}
\item Suppose that $x\sim y$ and $y<y'$.
\begin{itemize}
\item If $y'\models p$ and $x\models p$ then $x+3\sim y'$; if $y'\models p$ and $x\nmodels p$ then $x+3-\pi_3(x)\sim y'$.
\item If $y'\nmodels p$ and $x\models p$ then $x+1\sim y'$; if $y'\nmodels p$ and $x\nmodels p$ then $x+3\sim y'$.
\end{itemize}
\item Suppose that $x\sim y$ and $y'<y$.
\begin{itemize}
\item If $y'\models p$ and $x\models p$ then $x-3\sim y'$; if $y'\models p$ and $x\nmodels p$ then $x-\pi_3(x)\sim y'$.
\item If $y'\nmodels p$ and $x\models p$ then $x-1\sim y'$; if $y'\nmodels p$ and $x\nmodels p$ then $x-3\sim y'$.
\end{itemize}
\end{itemize}
\end{ex}

\begin{ex}
\begin{enumerate}
\item Note that $1\sim 2$ and that $1<2$. Thus, by one of the conditions of bisimilitions, there must exist an integer $k<1$ such that $k\sim 1$. This is not the case.
\item Note that $x\models GG\neg p$ precisely when $l\models\neg p$ for every $k>x$ and every $l>k$. Thus we see that $-1\models GG\neg p$, while $-2\nmodels GG\neg p$. Since for any bisimilation and any two worlds that are bisimilar, the theories of those worlds are the same and since $-1$ and $-2$ do not have the same theories, it follows that there is no bisimilation which renders $-1$ bisimilar to $-2$. In particular, the relation of the exercise does relate $-1$ to $-2$ and therefore it cannot be a bisimilation.
\end{enumerate}
\end{ex}
\end{document}
