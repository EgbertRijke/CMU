\documentclass{article}
\title{{\sc Philosophy Core Seminar Essay}\\ {\it \paperauthor~-- \papertitle}}
\author{Egbert Rijke}
\date\today

\usepackage{mathpazo}
\usepackage{amsmath,amssymb,amsthm}


\newcommand{\hwnumber}{6}
\newcommand{\topology}{{\mathcal{T}}}
\newcommand{\modalcirc}{{\bigcirc}}

\begin{document}
\maketitle

\begin{ex}
\begin{enumerate}
\item Every point $x$ which is not in $\closure{A}$ has an open neighborhood
$U_x$ such that $U_x\cap A=\varnothing$. Then $V:=\bigcup\{U\in\topology\mid
U\cap A=\varnothing\}$ contains every point which is not in the closure of $A$,
and none other, so $V=X\setminus\closure{A}$. Also, since $V$ is a union of
open sets, it is open. This shows that the complement of a closed set is open.

To show the converse, suppose $A$ is a set with the property that its complement
is open. Then for any $x\in X$, if every open neighborhood of $x$ intersects $A$, it follows that
$X\setminus A$ is not a neighborhood of $x$. Thus $A$ is the set of its own
closure points.

\item We use that \textsf{S5} is complete with respect to the class of all
topological spaces. We use the canonical space of maximal \textsf{S5}-consistent
sets of formulas, with basic open sets
\begin{equation*}
B_{\varphi}:=\{x_\Sigma\mid\nec\varphi\in\Sigma\}
\end{equation*}
for its topology. The valuation of this model is such that $x_\Sigma\models\varphi$
if and only if $\varphi\in\Sigma$. Consequently, we see that 
$B_\varphi=\lsem\nec\varphi\rsem$.

Our goal is to show that every basic open set is closed. In other words, we must
show that $\lsem\nec\varphi\rsem=\closure{\lsem\nec\varphi\rsem}$, where we note
that $\closure{\lsem\nec\varphi\rsem}=\lsem\pos\nec\varphi\rsem$. Since any set
is contained in its closure, the goal reduces to verifying the inclusion
$\lsem\pos\nec\varphi\rsem\subseteq\lsem\nec\varphi\rsem$. This holds if and only
if 
\begin{equation*}
\pos\nec\varphi\to\nec\varphi
\end{equation*}
is valid, which it is because it is an \textsf{S5}-theorem (by exercise 2(b) of
the second homework set).
\end{enumerate}
\end{ex}

\begin{ex}
\begin{enumerate}
\item For any set $A$ we have $A\subseteq\closure{A}$. In particular we have
$\{x\}\subseteq\closure{\{x\}}$ for any $x\in X$, so $R_\topology$ is reflexive.

Suppose that $x\mathbin{R_\topology} y$ and $y\mathbin{R_\topology} z$. Then
we have $y\in\closure{\{z\}}$, so by the minimality of closures we also have
$\closure{\{y\}}\subseteq\closure{\{z\}}$ and hence $x\in\closure{\{z\}}$. This
proves the transivity of $R_\topology$.
\item The relation $R_\topology$ needs not be symmetric. Consider the space
$X:=\{x,y\}$ with the Sierpinsky topology $\{\varnothing,\{x\},\{x,y\}\}$ on it.
Then $\closure{\{x\}}=X$ and $\closure{\{y\}}=\{y\}$. Thus, we see that
$y\mathbin{R_\topology} x$, while $\neg(x\mathbin{R_\topology}y)$. 
\item The only thing which differs in the two models is how the modal operators
are interpreted. Thus, in the proof by induction on modal formulas, 
we only have to show that for all $x\in X$
\begin{equation*}
((X,\topology,\lsem{-}\rsem),x)\models\pos\varphi
  \qquad
  \text{if and only if}
  \qquad
((X,R_\topology,\lsem{-}\rsem),x)\models\pos\varphi
\end{equation*}
provided that for all $x\in X$
\begin{equation*}
((X,\topology,\lsem{-}\rsem),x)\models\varphi
  \qquad
  \text{if and only if}
  \qquad
((X,R_\topology,\lsem{-}\rsem),x)\models\varphi
\end{equation*}
Note that $((X,\topology,\lsem{-}\rsem),x)\models\pos\varphi$ if and only if
$x\in\closure{\lsem\varphi\rsem}$, and that
$((X,R_\topology,\lsem{-}\rsem),x)\models\pos\varphi$ if and only if
there exists $y\in X$ such that $x\mathbin{R_\topology}y$ and
$((X,R_\topology,\lsem{-}\rsem),y)\models\varphi$. Thus, our goal is to show
that for all $x\in X$
\begin{equation*}
x\in\closure{\lsem\varphi\rsem}
  \quad
  \text{if and only if}
  \quad
\exists_{(y\in X)}(x\mathbin{R_\topology}y)\land (((X,R_\topology,\lsem{-}\rsem),y)\models\varphi).
\end{equation*}

The reverse direction is easy: suppose there exists $y\in X$ such that
$x\in\closure{\{y\}}$ and $((X,R_\topology,\lsem{-}\rsem),y)\models\varphi$,
then it follows that $x\in\closure{\{y\}}\subseteq\closure{\lsem\varphi\rsem}$.

For the direct direction we will use the condition that $X$ is an Alexandroff
space. Suppose that $x\in\closure{\lsem\varphi\rsem}$. There is a minimimal
open neighborhood $U_x$ of $x$, and we choose $y\in U_x\cap\lsem\varphi\rsem$.
Then it remains to show that $x\in\closure{\{y\}}$, but this is immediate from
the observation that every open neighborhood of $x$ contains $U_x$ and hence $y$.
\item Consider any infinite set $X$ with the cofinite topology, in which the closed
sets are either finite or $X$. Because singletons are finite and therefore
closed, the relation $R_\topology$ is the equality relation.
Let $x\in X$ and define $\lsem p\rsem:=X\setminus\{x\}$. Then $x\in\closure{\lsem p\rsem}$,
because the closure of any infinite set is $X$. However, $x$ is not equal to any
of the elements of $X\setminus\{x\}$, so none of the elements of $X\setminus\{x\}$
is $R_\topology$-accessible from $x$. This shows that
\begin{equation*}
((X,\topology,\lsem{-}\rsem),x)\models \pos p
\end{equation*}
while
\begin{equation*}
((X,R_\topology,\lsem{-}\rsem),x)\nmodels \pos p.
\end{equation*}
\end{enumerate}
\end{ex}

\begin{ex}
Before we begin with the solution, note that
\begin{align*}
\lsem\modalcirc\varphi\rsem
  & =
f^{-1}(\lsem\varphi\rsem).
\end{align*}
So we only need to reason with the semantic brackets, which is quicker in the
topological semantics.
\begin{enumerate}
\item Suppose $f:X\to X$ is locally constant. Then for
each $x\in X$, there is an open neighborhood $U$ of $x$ such that $U\subseteq f^{-1}(\{x\})$.
Therefore it follows that $f^{-1}(\{x\})$ is open for each $x$, and from this
it follows that $f^{-1}(A)$ is open for every $A\subseteq X$. In particular,
$f^{-1}(\lsem\varphi\rsem)$ is open for every modal formula $\varphi$. Now note
that $x\models\modalcirc\varphi$ if and only if $x\in f^{-1}(\lsem\varphi\rsem)$.
We conclude that the formula $\modalcirc\varphi\to\nec\modalcirc\varphi$ asserting
that $\lsem\modalcirc\varphi\rsem$ is always open, is valid.

We will show that $\modalcirc\varphi\to\nec\modalcirc\varphi$ defines the class of spaces
with a locally constant map. Suppose that the formula $\modalcirc\varphi\to\nec\modalcirc\varphi$
is valid. Then for any set $A$ we find that $f^{-1}(A)$ is open. 
In particular, $f^{-1}(\{f(x)\})$ is open for every $x\in X$, showing that $f$
is locally constant.
\item Suppose that $(X,\topology,f)$ is a dynamic space in which every element
of the form $f(x)$ is a sink. Then for any $x\in X$, we find
$f(x)\in\interior{f^{-1}(\{f(x)\})}$. As a consequence, we have
$B\cap f(X)\subseteq\interior{f^{-1}(B)}$ for any $B\subseteq X$. Then we also
see that $f^{-1}(B)=f^{-1}(B\cap f(X))\subseteq f^{-1}(\interior{f^{-1}(B)})$
for any $B\subseteq X$. In particular, we find that
\begin{equation*}
f^{-1}(\lsem\varphi\rsem)\subseteq f^{-1}(\interior{f^{-1}(\lsem\varphi\rsem)})
\end{equation*}
for any modal formula $\varphi$. Thus, we see that the modal formula
$\modalcirc\varphi\to\modalcirc\nec\modalcirc\varphi$ is valid in such dynamical
spaces.

Now suppose that the formula $\modalcirc\varphi\to\modalcirc\nec\modalcirc\varphi$
is valid in a dynamical space and let $x\in X$. Then we see that
\begin{equation*}
f^{-1}(\{f(x)\})\subseteq f^{-1}(\interior{f^{-1}(\{f(x)\})}).
\end{equation*}
It follows that $x\in f^{-1}(\interior{f^{-1}(\{f(x)\})})$, and thus that
$f(f(x))=f(x)$. Thus, we get that $f(x)\in f^{-1}(\{f(x)\})$ and as a consequence that
$f(x)=f(f(x))\in\interior{f^{-1}(\{f(x)\})}$. This proves the existence of an
open set $U$ containing $f(x)$ such that $f(y)=f(x)$ for all $y\in U$, i.e.~that
every element of the form $f(x)$ is a sink.
\end{enumerate}
\end{ex}

\begin{ex}
\begin{enumerate}
\item For any two distinct children $i$ and $j$, for all child $i$ knows,
child $j$ starts with no knowledge. So $i$ must assume that 
$\neg(m_1\lor\cdots\lor m_k)$ is possible for $j$. So child $i$ can only
derive that it has mud on his forehead if it sees no others with mud on their
forehead. Thus, this variant is only solvable for $k=1$.
\item This variant is solvable only when $k=1$ or $k=2$. 
\end{enumerate}
\end{ex}

\begin{ex}
\begin{enumerate}
\item Suppose that $w$ is a world so that every $C$-accessible world is a
$\varphi$-world. We have to show that $w\models E(\varphi\land C(\varphi))$.
In other words, we have to show that each $E$-accessible world from $w$ is
both a $\varphi$-world and a $C(\varphi)$ world.

Let $v$ be an $E$-accessible world from $w$. 
Since $C$ contains $E$, it follows that $v$ is also 
$C$-accessible. Therefore $v$ is a $\varphi$-world.

Suppose that $u$ is $C$-accessible from $v$. Since $C$ is transitive, we see
that $u$ is also $C$-accessible from $w$. Hence it is a $\varphi$-world. It
follows that $v$ is a $C(\varphi)$-world.
\item Let $\varphi$ and $\psi$ be formulas and assume that
$\models \varphi\to E(\varphi\land\psi)$. To show that $w\models\varphi\to C(\psi)$,
for every world $w$, let $w$ be a $\varphi$-world. In order to show that every
$C$-accessible world from $w$ is a $\psi$-world, we will show that every $C$-accessible
world from $w$ is both a $\varphi\land\psi$-world. We do this by showing
that $w\models E^n(\varphi\land\psi)$ for every $n\in\mathbb{N}$.

Note first that $w$ is a $(\varphi\to E(\varphi\land\psi))$-world. Since it is
assumed to be a $\varphi$-world, it follows that $w$ is also an $E(\varphi\land
\psi)$-world. This completes the base step of the induction proof.

Now suppose that $w$ is an $E^n(\varphi\land\psi)$-world and let $v$ be
$E^n$-accessible from $w$. It follows immediately that $v$ is a
$\varphi\land\psi$-world. Since $v$ is a $(\varphi\to E(\varphi\land\psi))$-world,
it follows that $v$ is a $E(\varphi\land\psi)$-world. This shows that every
$E^n$-accessible world from $w$ is an $E(\varphi\land\psi)$-world, so $w$ is
an $E^{n+1}(\varphi\land\psi)$-world,
completing the proof by induction.

\end{enumerate}
\end{ex}
\end{document}
