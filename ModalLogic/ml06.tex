\documentclass{article}
\title{Formal Methods homework set \hwnumber}
\author{Egbert Rijke}
\date\today

\usepackage{mathpazo}
\usepackage{amsmath,amsthm,amssymb}
\usepackage{xcolor}
\usepackage{comment}

\newcommand{\note}[1]{{\color{red}#1}}

\newcommand{\sigalg}{\mathcal}

\theoremstyle{definition}
\newtheorem{ex}{Exercise}



\newcommand{\hwnumber}{6}
\newcommand{\topology}{{\mathcal{T}}}

\begin{document}
\maketitle

\begin{ex}
\begin{enumerate}
\item Ever point $x$ which is not in $\closure{A}$ has an open neighborhood
$U_x$ such that $U_x\cap A=\varnothing$. Then $V:=\bigcup\{U\in\topology\mid
U\cap A=\varnothing\}$ contains every point which is not in the closure of $A$,
and none other, so $V=X\setminus\closure{A}$. Also, since $V$ is a union of
open sets, it is open. This shows that the complement of a closed set is open.

To show the converse, suppose $A$ is a set with the property that its complement
is open. Then for any $x\in X$, if every open neighborhood of $x$ intersects $A$, it follows that
$X\setminus A$ is not a neighborhood of $x$. Thus $A$ is the set of its own
closure points.

\item We use that \textsf{S5} is complete with respect to the class of all
topological spaces. In other words, for every non-theorem of \textsf{S5} there
is a space which refutes it. 
\end{enumerate}
\end{ex}

\begin{ex}
\begin{enumerate}
\item For any set $A$ we have $A\subseteq\closure{A}$. In particular we have
$\{x\}\subseteq\closure{\{x\}}$ for any $x\in X$, so $R_\topology$ is reflexive.

Suppose that $x\mathbin{R_\topology} y$ and $y\mathbin{R_\topology} z$. Then
we have $y\in\closure{\{z\}}$, so by the minimality of closures we also have
$\closure{\{y\}}\subseteq\closure{\{z\}}$ and hence $x\in\closure{\{z\}}$. This
proves the transivity of $R_\topology$.
\item The relation $R_\topology$ needs not be symmetric. Consider the space
$X:=\{x,y\}$ with the Sierpinsky topology $\{\varnothing,\{x\},\{x,y\}\}$ on it.
Then $\closure{\{x\}}=X$ and $\closure{\{y\}}=\{y\}$. Thus, we see that
$y\mathbin{R_\topology} x$, while $\neg(x\mathbin{R_\topology}y)$. 
\item The only thing which differs in the two models is how the modal operators
are interpreted. Thus, in the proof by induction on modal formulas, 
we only have to show that for all $x\in X$
\begin{equation*}
((X,\topology,\lsem{-}\rsem),x)\models\pos\varphi
  \qquad
  \text{if and only if}
  \qquad
((X,R_\topology,\lsem{-}\rsem),x)\models\pos\varphi
\end{equation*}
provided that for all $x\in X$
\begin{equation*}
((X,\topology,\lsem{-}\rsem),x)\models\varphi
  \qquad
  \text{if and only if}
  \qquad
((X,R_\topology,\lsem{-}\rsem),x)\models\varphi
\end{equation*}
Note that $((X,\topology,\lsem{-}\rsem),x)\models\pos\varphi$ if and only if
$x\in\closure{\lsem\varphi\rsem}$, and that
$((X,R_\topology,\lsem{-}\rsem),x)\models\pos\varphi$ if and only if
there exists $y\in X$ such that $x\mathbin{R_\topology}y$ and
$((X,R_\topology,\lsem{-}\rsem),y)\models\varphi$. Thus, our goal is to show
that for all $x\in X$
\begin{equation*}
x\in\closure{\lsem\varphi\rsem}
  \quad
  \text{if and only if}
  \quad
\exists_{(y\in X)}(x\mathbin{R_\topology}y)\land (((X,R_\topology,\lsem{-}\rsem),y)\models\varphi).
\end{equation*}

The reverse direction is easy: suppose there exists $y\in X$ such that
$x\in\closure{\{y\}}$ and $((X,R_\topology,\lsem{-}\rsem),y)\models\varphi$,
then it follows that $x\in\closure{\{y\}}\subseteq\closure{\lsem\varphi\rsem}$.

For the direct direction we will use the condition that $X$ is an Alexandroff
space. Suppose that $x\in\closure{\lsem\varphi\rsem}$. There is a minimimal
open neighborhood $U_x$ of $x$, and we choose $y\in U_x\cap\lsem\varphi\rsem$.
Then it remains to show that $x\in\closure{\{y\}}$, but this is immediate from
the observation that every open neighborhood of $x$ contains $U_x$ and hence $y$.
\item Consider any infinite set $X$ with the cofinite topology, in which the closed
sets are either finite or $X$. Because singletons are finite and therefore
closed, the relation $R_\topology$ is the equality relation.
Let $x\in X$ and define $\lsem p\rsem:=X\setminus\{x\}$. Then $x\in\closure{\lsem p\rsem}$,
because the closure of any infinite set is $X$. However, $x$ is not equal to any
of the elements of $X\setminus\{x\}$, so none of the elements of $X\setminus\{x\}$
is $R_\topology$-accessible from $x$. This shows that
\begin{equation*}
((X,\topology,\lsem{-}\rsem),x)\models \pos p
\end{equation*}
while
\begin{equation*}
((X,R_\topology,\lsem{-}\rsem),x)\nmodels \pos p.
\end{equation*}
\end{enumerate}
\end{ex}

\begin{ex}
\begin{enumerate}
\item Suppose $f:X\to X$ is locally constant and let $A\subseteq X$. 
\end{enumerate}
\end{ex}

\begin{ex}
Let $N_k:=\{0,\ldots,k-1\}$ and consider the frame 
\begin{equation*}
F:=(2^{N_k},R_0,\ldots,R_{k-1},E,C)
\end{equation*}
where $C=E^\ast$, $E=\bigcup_{i\in N_k}R_i$ and $f\mathbin{R_i}g$ if and only if 
$f|_{N_k\setminus\{i\}}=g|_{N_k\setminus\{i\}}$ for $f,g:N_k\to 2$.
Note that each $R_i$ is an equivalence relation. We also consider
the basic propositions $m_1,\ldots,m_k$ where $m_i$ is to be read
as `child $i$ has a muddy forehead'. The model we get from the
valuation given by $\lsem m_i\rsem:=\{f:N_k\to 2\mid f(i)=1\}$
then models the setup of the muddy children problem.

In the standard muddy children problem, it is furthermore given
that it is common knowledge that at least one of the children
has a muddy forehead, i.e.~
\begin{equation*}
C(m_1\lor\cdots\lor m_k).
\end{equation*}
\begin{enumerate}
\item 
\end{enumerate}
\end{ex}
\end{document}
