\documentclass{article}
\title{Formal Methods homework set \hwnumber}
\author{Egbert Rijke}
\date\today

\usepackage{mathpazo}
\usepackage{amsmath,amsthm,amssymb}
\usepackage{xcolor}
\usepackage{comment}

\newcommand{\note}[1]{{\color{red}#1}}

\newcommand{\sigalg}{\mathcal}

\theoremstyle{definition}
\newtheorem{ex}{Exercise}



\paperauthor{Frege}
\papertitle{On sense and reference}

\begin{document}
\maketitle

In explaining the meaning of a proper name or a sentence, Frege distinguishes between its sense and its reference. A proper name (or sign) is a description of an object and its reference is the actual object whereas its sense contains more cognitive information containing the mode of presentation. In Frege's words: `a proper name expresses its sense [and] stands for [...] its reference'. 

Thus it may happen that two proper names designate (i.e.~refer to) the same thing while they have different descriptions (i.e.~they differ in sense); the assertion that those two proper names are equal then contains the knowledge that the two constructions have the same result. When $a$ and $b$ are proper names, the assertion that $a$ is equal to $b$ designates the truth value of the fact that $a$ and $b$ designate the same object. Proper names for equal objects may always be replaced in a sentence without altering its meaning. The technique of replacing proper names with proper names designating the same thing is one of the cornerstones which Frege uses to test his further ideas on sense and reference. 

It is also important to note that a proper name may have sense without having an actual reference. For instance, the sign `the largest integer' has obvious sense, but fails to refer to any integer. It may also be the case, e.g.~when quoting, that the object of reference of a sign is itself a sign. Thus we see that in such cases we can evaluate the object of reference multiple times; in that case one speaks of the indirect reference. Frege further distinguishes the reference and sence of a sign from the associated idea. Since an idea is something which exists in a mind, ideas existing in different minds are not eligible to be equal, however similar they may appear; the notion of idea is the third level of difference of expressions.

The sentences which Frege considers contain thoughts and thus their designation is a truth value. Frege points out that words or parts of a sentence representing proper names may be replaced by proper names having the same designation, and although this will not alter the truth value of the sentence, it may alter the thought. Thus, thoughts are not the designations of such sentences, but rather their senses. Although the sense of a sentence seems more telling than its reference, the knowledged obtained by deriving the truth value is the reason why references of sentences are evaluated. A judgment is, from this perspective, the step from the thought to the truth value. 

The supposition that the reference of a sentence is a truth value implies furthermore that the truth value of a sentence remains unchanged when a proper name appearing in it is replaced with a proper name having the same designation. Also, all true sentences have the same reference, as do all false sentences. So designating a sentence forgets the specifics of it. Frege notes that neither mere thoughts nor truth values contain knowledge; knowledge is rather obtained from evaluating the truth value of a given sentence.

The situation is more complicated when one wants to replace a subsentence of a sentence with another subsentence having the same truth value. Here it is necessary to consider the form of the subsentence. For instance, a noun clause introduced by `that' refers to a thought rather that to the truth value of the thought. Then, the truth of the sentence does in fact not rely on the truth value of the thought expressed by the subsentence. Actually, in the case of an imperative or a question, such a sentence does not even have a reference, only a sense. Note that the reason that the subsentence in this class of examples refers to a thought is that an indirect reference is made. 

In conditional clauses (e.g.~`every natural number is the sum of four squares') it is possible to have an indefinite indicator occurring as well in the dependent clause. Such indefiniteness is used to increase the generality of the statement. The antecedent clause expressing the domain of the indefinite indicator does not designate a thought by itself in such cases. It is through these conditionals that we find the possibility that a subsentence expresses only a part of the thought and not a thought by itself. Thus we may not simply replace the subsentence by a sentence with the same truth value, since that would change the thought. 

a later mention in the sentence it does not. Thus also here such subsentences may not be replaces by subsentences designating equal truth values. To see how that goes in the case of sums of four squares, we see that the sentence expresses the thought that if $x$ is a natural number, then there are four squares whose sum is $x$.

Adjective clauses may be used in proper names to give more elaborite descriptions, although an adjective clause can never be a proper name by itself. An adjective clause does not have a thought as sense and neither does it have a truth value as its reference. Its sense is rather a part of the thought. 
\end{document}
