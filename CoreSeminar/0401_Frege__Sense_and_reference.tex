\documentclass{article}
\title{{\sc Philosophy Core Seminar Essay}\\ {\it \paperauthor~-- \papertitle}}
\author{Egbert Rijke}
\date\today

\usepackage{mathpazo}
\usepackage{amsmath,amssymb,amsthm}


\paperauthor{Frege}
\papertitle{Sense and reference}

\begin{document}
\maketitle

In explaining the meaning of a sign or a sentence, Frege distinguishes between its sense and its reference. A sign (or name) is a description of an object and its reference is the actual object whereas its sense contains more cognitive information containing the mode of presentation. Thus it may happen that two signs designate (i.e.~refer to) the same thing while they have different descriptions (i.e.~they differ in sense); the assertion that those two signs are equal then contains the knowledge that the two constructions have the same result. When $a$ and $b$ are names, the assertion that $a$ is equal to $b$ designates the truth value of the fact that $a$ and $b$ designate the same object. Names for equal objects may always be replaced in a sentence without altering its meaning. The technique of replacing signs with signs designating the same thing is one of the cornerstones which Frege uses to test his further ideas on sense and reference. 

It is also important to note that a sign may have sense without having an actual reference. For instance, the sign `the largest integer' has obvious sense, but fails to refer to any integer.

\end{document}
