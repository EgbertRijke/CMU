\documentclass{article}
\title{Formal Methods homework set \hwnumber}
\author{Egbert Rijke}
\date\today

\usepackage{mathpazo}
\usepackage{amsmath,amsthm,amssymb}
\usepackage{xcolor}
\usepackage{comment}

\newcommand{\note}[1]{{\color{red}#1}}

\newcommand{\sigalg}{\mathcal}

\theoremstyle{definition}
\newtheorem{ex}{Exercise}



\paperauthor{Russell}
\papertitle{On denoting}


\begin{document}
\maketitle

The main message of Russell's essay on denoting should have been a certain
phrase with the interaction between denoting phrases, propositions and meanings
as its meaning, accompanied with evidence for the proposition that this might
indeed be used effectively in an attempt to understand how we use language to 
gather knowledge. Indeed, the importance of denoting is that knowledge
is obtained through denoting and by means of denoting we may obtain knowledge
of objects of which we have no acquaintence. The basic idea is that denoting
phrases describe objects of our world including among other things the physical 
world, experiences, imaginations and abstractions; and that we use denoting 
phrases in propositions to make assertions about those objects.

In his theory meaning and denoting, Russell distinguishes two classes of
expressions about objects: that of denoting phrases and
propositions. To some of these, a meaning (or even multiple meanings) might be
assigned. As a means of summarizing his ideas of denoting phrases, let us try
to translate them into algebraic language. Thus, we have a set $D$ of denoting
phrases and a type $O$ of objects of the world (it is not clear that this type
should be a set) and meaning could be represented by a relation 
$M\hookrightarrow D\times O$. Every denoting phrase is also considered an
object, so there is also a monomorphism $D \hookrightarrow O$ which assigns to
every denoting phrase the object representing that denoting phrase. If $d$ is
a denoting phrase, then there are also denoting phrases $\mu(d)$ for `the meaning
of $d$' and $\delta(d)$ for the denotation of $d$. There are no further relations
on these functions $\mu,\delta:D\to D$; Russell stresses that it is not the case
that $\mu(d)=d$. 

In his theory of propositions, Russell proposes a system where every
proposition can be rewritten using
\begin{enumerate}
\item $C(\text{everything})$, which means that `$C(x)$ is always true'.
\item $C(\text{nothing})$, which means that ``$C(x)$ is false', is always true'.
\item $C(\text{something})$, which means that `it is false that `$C(x)$ is
false' is always true'. It goes unargued why `$C(x)$ is false' or `$x$ is
identical to $y$' do not appear in his list of propositions and the modern
convention is that $C(\text{something})$ would mean `there exists an $x$ such
that $C(x)$'. 
\end{enumerate}

An eminent feature of denoting phrases and propositions in Russell's theory is 
that apparently they have to be evaluated regardless of the context in which 
they are made. Russell gives the impression that the meaning of `the king of 
France' is a priory void, regardless of the fact that there has been a king of 
France and there might actually be at any moment someone who wishes to be 
referred to as the king of France (e.g.~because he has that role in another
context such as a story).

It is unclear how propositions relate to sentences (of a natural language). In
all the examples Russell provides, propositions are English sentences providing
a noun phrase and assigning a quality to it, e.g.~`the king of France is bald',
yet Russell is of the opinion that `the author of Waverley' is not
the subject of the sentence `the author of Waverley was a man'. Which position
he does take relative to linguistics and symbolic logic remains a unremarked
upon and this makes it hard to evaluate the reach of his claims. 

\end{document}
