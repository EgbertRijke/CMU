\documentclass{article}

\title{{\sc Philosophy Core Seminar Essay}\\ {\it \paperauthor~-- \papertitle}}
\author{Egbert Rijke}
\date\today

\usepackage{mathpazo}
\usepackage{amsmath,amssymb,amsthm}


\paperauthor{Davidson}
\papertitle{The logical form of action sentences}

\begin{document}
\maketitle

The goal of Davidson is to provide a coherent theory of action sentences which
describes how to bring an action sentence to its logical form. Obviously,
rewriting an action sentence to its logical form should preserve its truth
evaluation. Moreover, the theory should adequately address relative 
qualifications (such as doing something slowly), supposed intention and
both active and passive agents. Davidson illustrates the importance of adressing
each of these attributes by pointing out the inadequacies in previous
proposals on how to bring action sentences to their logical form.

The first attempt to formulate a means to rewrite action sentences to their
logical form mentioned in the article is Kenny's, with some minor variations of
Chisholm and von Wright. Kenny proposes that `$x$ did $p$' may be replaced by
`$x$ brings it about that $p^*$', where $p^*$ is the result of doing $p$ (the
notation $p^*$ is not used in the article). Since this replaces
the act of doing $p$ by the result of doing $p$, information about doing $p$
is necessarily lost. For instance, since `$x$ brings about $p^*$' asserts only 
that $x$ has caused that the final state of the event is $p^*$, it might have
been the case that $x$ has ordered $y$ to do $p$. 

Von Wright's attempt is to also include the state
\emph{before} doing $p$ in the replaced form. Also in von Wrights proposal, the
information of the action sentence `$x$ did $p$' is not fully covered. Moreover,
Davidson points out that action sentences often do not include information
about the initial state. While this suffers from the same fallacies as Kenny's
approach, von Wright also brings up the distinction between generic and
indivicual propositions about events.

The direction in which Davidson is heading is closer to that of Reichenbach. He
introduces an existential quantification in order to rewrite an action sentence,
although he does not consider the result to be in logical form because he would
already consider the original action sentence logical. Davidson then proposes
a strange way of translating the existential statement, where the variable
presumably ranges over all possible events, to a statement in natural language:
$(\exists x)(x\text{ consists in the fact that }\phi)$ would, according to
Davidson, be translated into `$\phi^*$ took place'. In Davidson's example where
the action sentence is `Amundsen flew to the North Pole', $\phi^*$ would be
`A flight by Amundsen to the North Pole'. Davidson changes the
subject of the sentence and blames Reichenbach for it.

Davidson proposes to assign a predicate to any verb action. According to his
proposal, the logical
form of an action sentence would state the existence of an event $x$ such that
$x$ is $\phi$, where $\phi$ is to be a description of the action event under
consideration. 

I contest, as Reichenbach did, that the existential quantification over the class of all events is 
really necessary. In the propositions as types paradigm, any statement of the 
form `there exists $x$ of type $X$ such that $x$ is identical to $a$' is
contractible and does therefore not provide any more information than just 
providing the term $a$ directly. Thus it seems that the Reichenbach construction
is overly cautious: if we are to add predicates, we might as well introduce the
proposition
\begin{center}
$\text{Kicked}(\text{Shem},\text{Shaun})$
\end{center}
instead of $(\exists x)\,\text{Kicked}(\text{Shem},\text{Shaun},x)$ to denote the event that
Shem kicked Shaun; and likewise we might take
\begin{center}
$\text{To}(\text{the Morning Star},
\text{FlewSomething}(\text{my spaceship},\text{I}))$
\end{center}
rather than
\begin{center}
$(\exists x) (\text{Flew}(\text{I},\text{my spaceship},x)\ \&\ \text{To}(\text{the
Morning Star},x))$.
\end{center}
It is furthermore plain from these modifications how to unfold the formal
sentence into a sentence of (any) natural language without an over-use of
conjunctions and without a superficial identity statement.

Davidson does not make any extra structure such predicates might come with 
explicit. For instance, action events do not happen in isolation but might be 
concatenated. Similarly,
one might expect that in the theory of action sentences (which are after all
denoting action events) there would be such a similar structure of combining
sentences and hence predicates. Computer scientists have encountered action 
events too and use monads to formalize them. It would go to far to explore here
such a possibility for action sentences, but I would be surprised if merely
introducing a predicate for each action sentence is the end of the story.

Finally, while Davidson published his article 30 years after Strawson published
his article \emph{On Referring}, Davidson makes no mention of his ideas. In
particular, he does not make a distinction between an action sentence, its uses
and its utterings and neither does he indicate why he does not do so. Yet
Strawson's ideas seem relevant here too. While Davidson would argue that the
logical form of an action sentence is an existential proposition, Strawson would
instead argue that a \emph{use} of an action sentence would attempt to make a
reference to an action event. This would provide for a second argument against
the Russellian use of the existential quantification.

\end{document}
