\documentclass{article}

\title{{\sc Philosophy Core Seminar Essay}\\ {\it \paperauthor~-- \papertitle}}
\author{Egbert Rijke}
\date\today

\usepackage{mathpazo}
\usepackage{amsmath,amssymb,amsthm}


\paperauthor{Davidson}
\papertitle{The logical form of action sentences}

\begin{document}
\maketitle

The goal of Davidson is to provide a coherent theory of action sentences which
describes how to bring an action sentence to its logical form. Obviously,
rewriting an action sentence to its logical form should preserve its truth
evaluation. Moreover, the theory should adequately address relative 
qualifications (such as doing something slowly), supposed intention and
both active and passive agents. Davidson illustrates the importance of adressing
each of these attributes by pointing out the inadequacies in previous
proposals on how to bring action sentences to their logical form.

The first attempt to formulate a means to rewrite action sentences to their
logical form mentioned in the article is Kenny's, with some minor variations of
Chisholm. Kenny proposes that `$x$ did $p$' may be replaced by
`$x$ brings it about that $p^*$', where $p^*$ is the result of doing $p$ (the
notation $p^*$ is not used in the article). Since this replaces
the act of doing $p$ by the result of doing $p$, information about doing $p$
is necessarily lost. For instance, since `$x$ brings about $p^*$' asserts only 
that $x$ has caused that the final state of the event is $p^*$, it might have
been the case that $x$ has ordered $y$ to do $p$. 

Von Wright's attempt is to also include the state
\emph{before} doing $p$ in the replaced form. Also in von Wrights proposal, the
information of the action sentence `$x$ did $p$' is not fully covered. Moreover,
Davidson points out that action sentences often do not include information
about the initial state

Davidson proposes to assign a predicate to any verb action

Davidson does not include any extra structure such predicates might come with.
For instance, action events do not happen in isolation but might be 
concatenated. Similarly,
one might expect that in the theory of action sentences (which are after all
denoting action events) there would be such a similar structure of combining
sentences and hence predicates. Computer scientists have encountered action 
events too and use monads to formalize them. 

\end{document}
