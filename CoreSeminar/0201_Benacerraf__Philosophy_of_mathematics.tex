\documentclass{article}
\title{Formal Methods homework set \hwnumber}
\author{Egbert Rijke}
\date\today

\usepackage{mathpazo}
\usepackage{amsmath,amsthm,amssymb}
\usepackage{xcolor}
\usepackage{comment}

\newcommand{\note}[1]{{\color{red}#1}}

\newcommand{\sigalg}{\mathcal}

\theoremstyle{definition}
\newtheorem{ex}{Exercise}



\paperauthor{Benacerraf and Putnam}
\papertitle{Philosophy of mathematics}


\begin{document}
\maketitle

The question about the epistemological content of (classical) mathematics is one that 
concerns the very foundations of mathematics; the mathematical implications of
possible answers move the question beyond mere philosophy. Three points of view
are presented: that of the logicist, the intuitionist and the formalist.
The role of the theories of integers and of real numbers is principal in all
points of view. Although the derivation of the system of integers presents no
real problems in either approach, the construction of the real numbers presents
problems which neither of the three can fully overcome.

The logicist point of view, which has been held most noteably by Russell and
Whitehead, is that all of mathematics is reducible to logic. While one might
argue that definitions and theorems should really be synonyms (as is currently the practice
in type theory), Carnap splits the logicists thesis into two parts: (1) the
concepts of mathematics can be derived from logical concepts through explicit
definitions; and (2) the theorems of mathematics can be derived from logical
axioms through purely logical deduction. The system of propositional calculus
should, according to the logicist, be enough to provide the means for the
construction of mathematics. Then, every provable mathematical sentence can be
translated into a sentence using only primitive logical symbols, which is then
derivable according to the rules of logic.

However, some of the classical theorems of arithmetic and set theory require
further axioms besides the logical ones, such as the axioms of infinity or the
axiom of choice, because logic only deals with possible entities and cannot
make assertions of the existence of things, according to Carnap. While this
might be the case for propositional calculus, on which set theory is based,
the propositions as types philosophy provides systematic means for including
inductive types and thereby avoiding the need for having an ad hoc axiom of
infinity. In the view of modern type theorists, the $W$-type constructor for
inductive types is just another of the primitive constants, which is overlooked
when attention is restricted to propositional calculus.

The construction of the real numbers by Dedekind presents a real number as a
cut of rational numbers. A Dedekind cut consists of a left and a right class of
rational numbers such that each of the rationals belongs to one of them; if
a rational is less than a rational in the left class then it is in the left
class and if it is bigger than something in the right class it is in the right.
This is an impredicative definition and to overcome the difficulties that come
with impredicativity, Russell has come up with the simple theory of types where
a stratification of classes is used to prevent one from deriving contradictions
of the nature of Russell's paradox. 

Intuitionists see mathematics as a free activity of thought. In their view,
attributing an existence, such as a choice function, independent of our thought
infers with the guarantee that all mathematical constructions are actual constructions
of the mind, which they aim for. In a line of thought similar to that of Russell's
stratification of type theory, intuitionists consider impredicative definitions
impossible because according to them it is self-evident that in the construction
of a species only previously defined objects may have an appearence. Brouwer has
introduced a theory of spreads of real numbers, where spreads are choice-sequences
of rationals following certain axioms to follow the Dedekind construction. I found
the section on spreads not quite clarifying, so I will not comment on that any further.

Hilbert's aim, presenting the formalist's point of view, was to prove the validity
of mathematics by investigating methods of proof. Mathematics is viewed as a
combinatorial game played with the primitive symbols of logic, and the golden
rule is to find a finitary combinatorial way to construct proofs. Thus, the
program of Hilbert lead to find all the symbols used in logic and mathematics;
to characterize unambiguously all the combinations of these symbols which
represent statements classified as meaningful; supply a construction procedure
to give precise meaning to the act of proving; and finally to show that the
provable statements of classical mathematics indeed have a translation and a
proof in this game. However, proving the consistency of mathematics this way
has proved to be a daunting task, at best, especially in the view of G\"odel's
incompleteness theorems.

It might be worthwile to note that in Voevodsky's program of the Univalent Foundations of Mathematics, another solution for the construction of the real numbers has been envisioned. This construction is of an inductive kind and exploits the possibility of freely adding propositional equality between terms (here, free must be taken as in free algebras). In this construction the impredicativity of Dedekind's real numbers is avoided and Cauchy completeness is guaranteed without the need for an axiom of choice; thus the construction should be satisfactory from a constructivist point of view. The construction goes very much like that of the Cauchy real numbers, except that not only Cauchy sequences of rational numbers are considered in the definition, but Cauchy sequences of real numbers alike. For this, it is needed that an $\epsilon$-proximity relation is generated for every positive rational number $\epsilon$, \emph{within} the definition of the real numbers. While Heyting argues in his article that `\emph{absolutely no ordering relations have been defined between [Euler's constant] and the rational numbers}', such an ordering is built-in from the start in the univalent real numbers. To do this, an inductive-inductive approach has to be taken, but this does not imply a non-constructive nature of the construction.
\end{document}
