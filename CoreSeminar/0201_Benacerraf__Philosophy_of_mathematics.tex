\documentclass{article}
\title{{\sc Philosophy Core Seminar Essay}\\ {\it \paperauthor~-- \papertitle}}
\author{Egbert Rijke}
\date\today

\usepackage{mathpazo}
\usepackage{amsmath,amssymb,amsthm}


\paperauthor{Benacerraf and Putnam}
\papertitle{Philosophy of mathematics}


\begin{document}
\maketitle

The question about the epistemological content of mathematics is one that concerns the very foundations of mathematics.

The logicist point of view is that all of mathematics is reducible to logic.

It might be worthwile to note that in Voevodsky's program of the Univalent Foundations of Mathematics, another solution for the construction of the real numbers has been envisioned. This construction is of an inductive kind and exploits the possibility of freely adding propositional equality between terms (here, free must be taken as in free algebras). In this construction the impredicativity of Dedekind's real numbers is avoided and Cauchy completeness is guaranteed without the need for an axiom of choice; thus the construction should be satisfactory from a constructivist point of view. The construction goes very much like that of the Cauchy real numbers, except that not only Cauchy sequences of rational numbers are considered in the definition, but Cauchy sequences of real numbers alike. For this, it is needed that an $\epsilon$-proximity relation is generated for every positive rational number $\epsilon$, \emph{within} the definition of the real numbers. While Heyting argues in his article that `\emph{absolutely no ordering relations have been defined between [Euler's constant] and the rational numbers}', such an ordering is built-in from the start in the univalent real numbers. To do this, an inductive-inductive approach has to be taken, but this does not imply a non-constructive nature of the construction.

\end{document}
