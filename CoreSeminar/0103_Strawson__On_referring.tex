\documentclass{article}

\title{Formal Methods homework set \hwnumber}
\author{Egbert Rijke}
\date\today

\usepackage{mathpazo}
\usepackage{amsmath,amsthm,amssymb}
\usepackage{xcolor}
\usepackage{comment}

\newcommand{\note}[1]{{\color{red}#1}}

\newcommand{\sigalg}{\mathcal}

\theoremstyle{definition}
\newtheorem{ex}{Exercise}



\paperauthor{Strawson}
\papertitle{On referring}

\begin{document}
\maketitle

Strawson argues that Russel's theory of Descriptions is not in agreement with
how people use language to state propositions with reference to individuals
or objects. He points out that for each expression (in his usage: phrase) or 
sentence we should distinguish
\begin{enumerate}
\item the expression or sentence;
\item uses of the expression or sentence;
\item utterings of the expression or sentence.
\end{enumerate}
For instance, two different
people who use the same expression `I' use it to refer to different individuals
(both to themselves). 
According to Strawson, the meaning of an expression consist of general 
directions to use it to refer to an individual or thing; the meaning of a
sentence consists of general directions to use it to make true or false 
assertions. It is not a sentence which can be true or false, it is rather the
use of a sentence which may result in a true or false statement. It may also be
the case that a particular use of an expression fails to make any reference; the
use of the sentence then refrains from having meaning. With this, Strawson opens
the possibility for uttered assertions without any truth value, e.g.~someone
uttering that the present king of France is wise, while France is a republic.
Indeed, according to Strawson, expressions of the form `the so-and-so' are used
to make a reference to a particular person or thing which is a so-and-so and
not, as Russell proposed, as unique existential propositions.

Strawson also argues that the significance of an expression or sentence lies
in the possibility for someone ever to use it to describe an actual person or 
thing (in the case of an expression) or to make a true or false assertion, and
thus the notion of significance is independent on any particular occasion.

Strawson's ideas seem to effectively address how meaning relates to assertions
people make about persons or things in order to express properties of them.
However, action sentences are not in the picture here while they could perfectly
be a subject of his theory. He ends furthermore with one little flaw: universal
quantification does not imply existential quantification. 

\end{document}
