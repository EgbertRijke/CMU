\documentclass{article}
\title{{\sc Philosophy Core Seminar Essay}\\ {\it \paperauthor~-- \papertitle}}
\author{Egbert Rijke}
\date\today

\usepackage{mathpazo}
\usepackage{amsmath,amssymb,amsthm}


\paperauthor{Russell}
\papertitle{Our knowledge of the external world}


\begin{document}
\maketitle

Russell discusses the aim and scope of philosophy, provided it were to be a science. Experimental data not agreeing with established philosophical ideas, especially about the a priori, have impacted hard on the subject and a discussion on the methods and pretensions were therefore needed at the time he gave his lectures. Russell dispels both the classical tradition and evolutionism and proposes logical atomism as the favorable approach to modern philosophy. Language has to be used in a logically precise sense to make assertions about objects, much in the same way as in his theory explained in On Denoting. 

I have not seen, or I have failed to notice, a precise statement about the aim and scope of modern philosophy and leaves the impression that philosophy is still to be omnipresent in all areas where knowledge is acquired. Yet the methods of the philosopher are not envisioned to be the same as the methods of the experimenter or the problem solver, for these are never discussed. Russell still envisions that from a body of common knowledge we may obtain through logical analysis and reduction a system of interconnected propositions which are arranged in deductive chains and which is wholly free from logical redundancy, wholly precise and as simple as is logically compatible with the body of knowledge started with. This leaves little to be improved and does not regard the resources of the person who is to act based on the given knowledge (he might have to be quick in order for his data to remain relevant), and this desire seems therefore impractical or even unfeasible, even though he allows for uncertainty through the logical compatibility. 

I find the writings of Carnap pretty much unintelligable, but that probably says more about me than about Carnap. He talks of relations and aims to apply his very abstract theory to obtaining knowledge of the external world, but it remains unclear to me how he sees his theory used to obtain knowledge of the external world.
\end{document}
