\documentclass{article}
\title{{\sc Philosophy Core Seminar Essay}\\ {\it \paperauthor~-- \papertitle}}
\author{Egbert Rijke}
\date\today

\usepackage{mathpazo}
\usepackage{amsmath,amssymb,amsthm}


\paperauthor{Russell}
\papertitle{Our knowledge of the external world}

\date{September 16, 2014}


\begin{document}
\maketitle

Russell discusses the aim and scope of philosophy, provided it were to be a science. Experimental data not agreeing with established philosophical ideas, especially about the a priori, have impacted hard on the subject and a discussion on the methods and pretensions were therefore needed at the time he gave his lectures. Russell dispels both the classical tradition and evolutionism and proposes logical atomism as the favorable approach to modern philosophy. Language has to be used in a logically precise sense to make assertions about objects, much in the same way as in his theory explained in On Denoting. 

I have not seen, or I have failed to notice, a precise statement about the aim and scope of modern philosophy and leaves the impression that philosophy is still to be omnipresent in all areas where knowledge is acquired. Yet the methods of the philosopher are not envisioned to be the same as the methods of the experimenter or the problem solver, for these are never discussed. Russell still envisions that from a body of common knowledge we may obtain through logical analysis and reduction a system of interconnected propositions which are arranged in deductive chains and which is wholly free from logical redundancy, wholly precise and as simple as is logically compatible with the body of knowledge started with. This leaves little to be improved and does not regard the resources of the person who is to act based on the given knowledge (he might have to be quick in order for his data to remain relevant), and this desire seems therefore impractical or even unfeasible, even though he allows for uncertainty through the logical compatibility. 

In his theory of continuity, Russell criticizes and questions ideas about the space-time continuoum. Verifying that the space-time is in fact a continuum is not possible, for it would require us to consider arbitrarily small segments of the space-time fabric. But even though it may not be derived from our observations that space-time is a continuum, Russell does not consider that the continuum of points by which we model is very effective and practical for obtaining knowledge, making predictions and engeneering our environment. Nonetheless, the proposition that space-time is a continuum remains logically compatible with the observations. 

This is, however, unsure. Both the theories of general relativity and of quantum physics have been extensively verified experimentally, both within their scopes. General relativity at the astronomical scale and quantum mechanics at the tiniest scales. But much of the research in modern theoretical physics aim to resolve a fundamental conflict: when the theories of gravity and quantum fields meet, a wildly fluctuating space-time fabric is predicted. Thus it is not at all sure that we indeed observe a continuum. 

The point of Russell was, however, that a theory about the external world should be in agreement with the body of knowledge that is obtained through experiments or, in his words, sensations, and should be aware of the assumptions it makes that do not follow logically from the observations. Logical inference -- with experimental data as the starting point -- in a language where propositions about the external world (the constituents of the proposition) are given a precise logical form is the means by which we can most securely increase our body of knowledge, and this is the view he propagates.
\end{document}
