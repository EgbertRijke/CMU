\documentclass{article}
\title{{\sc Understanding proofs}\\{\footnotesize Summary of Avigad's paper}}
\author{Egbert Rijke}
\date\today

\usepackage{microtype}
\usepackage{amsmath,amssymb,amsthm}
\usepackage{etoolbox}

\AtBeginEnvironment{quote}{\sl}

\begin{document}
\maketitle

\hfill
\begin{minipage}{.4\textwidth}
\sl
Concepts lead us to make investigations; are the expressions of our interest,
and direct our interest.
\end{minipage}

\vspace{3\baselineskip}

The goal of the paper is to describe what is expected of a theory of 
mathematical understanding, and to identify the appropriate support for a claim
to mathematical knowledge. The notion of a mathematical proof needs to be
clarified, and it must be explained how such proofs are capable of providing
appropriate mathematical knowledge. 

In particular, it should be explained what is 
involved with understanding a mathematical theorem, definition, proof, a theory,
a method or an algorithm. Avigad considers two sorts of agents to model 
understanding behavior: the human and the computer. Automated reasoning is 
supposed to model the kind understanding needed in the verification of a 
theorem.

There is an intuitive difference between knowing that a mathematical claim is
true, and understanding why the claim is true. the words definition and concept
have different connotations. The fact that there is a gap between knowledge and
understanding is made pointedly clear by the fact that one often finds dozens
of published proofs of a theorem in the literature, all of which are deemed
important contributions, even after the first one has been accepted as correct.

\begin{quote}
Discovery consists precisely in not constructing useless combinations, but in
constructing those that are useful, which are an infinitely small minority.
Discovery is discernment, selection.
\end{quote}

Avigad presents the following metaphor to clarify the role of understanding:
Mathematics presents us with a complex network of roads; understanding
helps us navigate them, and find the way to our destination. Mathematics
presents us with a combinatorial explosion of options; understanding helps us
sift through them, and pick out the ones that are worth pursuing. Without
understanding, we are lost in confusion, wandering blindly, unable to cope.
When we do mathematics, we are like Melville's sailors, swimming in a vast
expanse. Just as the sailors cling to sides of their ship, we rely on our
understanding to guide us and support us.

Ascriptions of understanding are best understood in terms of the possession of
certain abilities, and it is an important philosophical task to try to
characterize the relevant abilities in sufficiently restricted contexts in which
such ascriptions are made. One way of clarifying an ascription of understanding
is by specifying some of the abilities that I take such an understanding to
encompass. When we talk about understanding, we are invariably talking about
the ability, or capacity, to do something such as to solve a problem, to choose
an appropriate strategy, to discover a proof, to discern a fruitful definition
from alternatives or the ability to apply a concept efficaciously. When we say
that someone understands we simply mean that they possess the relevant
abilities. Thus, it is often fruitful to analyze understanding in terms of the
possession of abilities. The ability to determine whether a proof is correct
is fundamental to mathematics, and the standard theory has a lot to say as to
what that ability amounts to. Verifying correctness is only a small part of
understanding a proof, and we commonly speak of understanding other sorts of
things as well. Thus we need to describe and analyze the interrelated network
of abilities which constitute the practice of mathematics, in relation to fields
of inquiry that rely, either implicitly or explicitly, on models of that
practice.

There seems to be some sort of reciprocal relationship between mathematics and 
understanding. Understanding is clearly needed to carry out certain mathematical
tasks, but although understanding can explain the ability to carry out a task
successfully, isn't it a mistake to conflate the two? We should expect a
philosophical theory of understanding to provide a characterization of
mathematical understanding that is consistent with, but independent of, our
subjective experience.

The idea to ascribe understanding in terms of the possession of certain abilities
accords well with Wittgenstein's views on language. An important segment of
Wittgenstein's Philosophical Investigations explores what it means to follow
a rule, as well as related notions, like obeying a command or using a formula
correctly. His analysis shows that,
from a certain point philosophical perspective, it is fruitless to hope for a 
certain type of explanation of the meaning of judgments of understanding.
Nor is it necessary: there are philosophical gains to be had by exploring the
relationships between such fundamental judgments, and unraveling problems that
arise from confused or misguided applications of the terms. Understanding is
only made manifest in an array of uses. The common philosophical tendency is
therefore to distinguish the two, and take understanding to be `possession' of
a meaning that somehow `determines' the appropriate usage.

If our goal is to explain what it means to say that someone has understood a
particular word, formula, or command, we simply need to describe the 
circumstances under which we are willing to make this assertion. Thus, from a
Wittgensteinian perspective, the philosopher's task is not to explain the
feeling of having understood, or any underlying mental or physical processes.
The challenge is rather to clarify the circumstances under which we make our
ascriptions.

We should not be distressed by the fact that our ascriptions of understanding
are fallible. The aim of the investigations is to shape the way we think about
language and thought. here, I have proposed that this world view is relevant to
the way we think about mathematical understanding. When it comes to the
philosophy of mathematics, I believe it is also fruitful to explore what we take
to constitute appropriate behavior, even in situations where we take a goal or
a standard of correctness to be fixed and unproblematic. The claim I am making
here is simply that the terrain we are describing is best viewed as a network
of abilities, or mechanisms and capacities for thought.

This stands in contrast to the traditional view of mathematics as a collection
of definitions and theorems. For Wittgenstein, a proposition is not just an
object of knowledge, but rather something that shapes our behavior. It is our
acting which lies at the bottom of the language game.

This way of thinking challenges us to view mathematics in dynamic terms, not as
a body of knowledge, but rather as a complex system that guides our thoughts
and actions.

A theory of mathematical understanding should be a theory of mathematical
abilities:
\begin{itemize}
\item The ability to respond to challenges as to the correctness of the proof,
and fill in details and justify inferences at a skeptic's request;
\item The ability to give a high-level outline, or overview of the proof;
\item The ability to cast the proof in different terms, say, eliminating or
adding abstract terminology;
\item The ability to indicate `key' or novel points in the argument, and
separate them from the steps that are straightforward;
\item The ability to motivate the proof. That is, to explain why certain steps
are natural, or to be expected;
\item The ability to give natural examples of the various phenomena described in
the proof;
\item The ability to indicate where in the proof certain of the theorem's
hypotheses are needed, and perhaps, to provide counterexamples that show what
goes wrong when various hypotheses are omitted;
\item The ability to view the proof in terms of a parallel development, for
example, as a generalization or adaption of a well-known proof of a simpler
theorem;
\item The ability to offer generalizations, or, to suggest an interesting
weakening of the conclusion that can be obtained with a corresponding of the
hypotheses;
\item The ability to calculate a particular quantity, or to provide an explicit
description of an object, whose existence is guaranteed by the theorem;
\item The ability to provide a diagram representing some of the data in the
proof, or to relate the proof to a particular diagram;
\item \ldots
\end{itemize}
The philosophical challenge is to characterize these abilities with clarity and
precision, and fit them into a structured and informative theory.

Our analysis entails that understanding only becomes manifests in an agent's
behavior across a range of contexts, and we seem to have come dangerously close
to identifying understanding with the class of relevant behaviors. Such a
dispositional or behavioral account of understanding has famously been put
forth by Gilbert Ryle as part of a more general philosophy of mind. Since Ryle's
approach is commonly viewed as having failed, it is worth reviewing some of the
usual criticisms to see what bearing they have on the more specific issues 
addressed here.

Ryle intended his dispositional theory to account for ascriptions of a variety
of mental states, including things like belief, desire, intent, and so on.
We may have concerns that there are situations under which it makes sense
to say that someone understands a proof, but is unable to exhibit the expected
behaviors. But the examples that come to mind are contrived, and it does not
seem unreasonable to declare these outside the scope of a suitably focused
theory of mathematical understanding. If we view at least the outward signs of
mathematical activity as essentially linguistic, it seems reasonable to take
verbal and written communication as reliable correlates of understanding.

These concerns translate into doubts that one can adequately characterie the
behaviors that warrant attributions of understanding. But the problem can be
mitigated by limiting the scope. We would like our theory to help explain why
certain proofs are preferred in contemporary mathematics, why certain historical
developments were viewed as advances, or why certain expository practices yield
desired results. Moving from a general theory of mind to a more specific
theory of mathematical understanding gives us a great latitude in bracketing
issues that we take to fall outside our scope. We should be able to screen off
extraneous beliefs and desires, and assume nothing about an agent's intent
beyond the intent to perform mathematically.

Perhaps the most compelling criticism of a dispositional account is that even
if we could characterize the behaviors that are correlated with the mental states
under consideration, 
the theory which identifies the mental states with the associated behaviors
does not provide some sort of causal explanation that tells us how intelligent
and intentional behavior is brought about. We would like a theory that explains
how a proper understanding enables one to function mathematically. Insofar as
our theory is to be relevant to mathematical exposition and pedagogy, we would
expect it not only to characterize the outward signs of mathematical 
understanding, but also provide some hints as to how they can be encouraged and
taught. Similarly, a theory of mathematical understanding should be of service 
to computer scientists trying to write software that exhibits various types of
competent mathematical behavior. In short, we might expect a useful 
philosophical theory to both clarify and characterize desired behaviors, as well
as to provide some guidance in bringing them about.

Our theory of mathematical abilities need not degenerate to a laundry list of
behavioral cues. The abilities we describe interact in complex ways and will 
not always be cast in terms of behavioral manifestations.

Informal notions of understanding are useful in the implementation of proof
verification programs, a.k.a.~(interactive) proof assistants. 
A user's interaction with a proof assistant can be seen as an attempt to provide
the computer with enough information to see that there is a formal axiomatic
proof of the purported theorem. Alternatively, the formal proof scripts that
are given to the computer can be viewed as providing instructions to how to
find a formal axiomatic derivation. The task of the computational proof assistant
is to use these scripts to construct such a derivation. When it has done so,
the system indicates that it has `understood' the user's proof by certifying
the theorem as correct.

Without the theory of 
understanding, writing a proof in a proof assistant becomes an excessively
daunting task, which suffers from problems like
\begin{enumerate}
\item Combinatorial explosion of the number of cases: Computers
can search exhaustively for an axiomatic derivation of the desired inference,
but blind search does not get very far. The space of possibilities grows
exponentially. Users are forced to verify a large amount of possibilities in 
case distinctions.
\item Proving an existential conclusion, or using a universal hypothesis, can
require finding appropriate instances. There may be infinitely many terms to
consider. Among the strategies that have been employed in the implementation
of proof assistants are: defer choosing terms as long as possible, using
Skolem functions and a process called unification. Even in the presence of such
methods, there will be choices to be made, resulting in explosion.
\item There is a large class of inferences that require very little effort on
our part, but are beyond the means of current verification technology.
Commonly, inferences require background facts and theorems
that are left implicit. Each discipline typically has its own bag of tricks. 
In this case, the challenge is to determine which
external facts and theorems need to be imported to the local hypotheses. The
range of options once again render the problem intractable. 
\item Irrelevant information: Users are forced to keep track of irrelevant
information, where every detail is spelled out precisely in such a way that
pattern matching against a manageable list of precisely specified rules, whereas
in textbooks, readers are called upon their ability to see that each of the
successive steps is warranted.
\item Even though in restricted domains decision procedures might exist, we
know as a consequence of the incompleteness theorem that decidability of
propositions is the exception rather than the rule. Where they exist, they tend
to be inefficient or even infeasible. There are many instances where a short
derivation is possible, but where a general decision procedure follows a more
circuitous route.
\end{enumerate}
The flip explanation as to why competent mathematicians succeed where computers
fail is simply that mathematicians understand, while computers don't. At a
suitable level of abstraction, an account of how this understanding works should
not only explain the efficacy of our own mathematical practices, but also help
us fashion computational systems that share in the succes.

A theory of mathematical understanding should clarify the structural aspects of
mathematics that characterize successful performance for agents of any sort.
The theory should differentiate good teaching practice from good programming
practice, the theory should be relativized to the particuliarities of the relevant
class of agents. We see nothing ruling out that a general philosophical theory
can help explain what makes it possible for either type of agent to understand
a mathematical proof.

Computers are good at
\begin{itemize}
\item performing an exhaustive search. Some strategies that may speed up the 
search procedure include: backchaining, forward search and combining methods.
\item carrying out symbolic calculations and deriving equalities by means of 
normalization.
\end{itemize} 
Often mathematical facts are verified by combining methods from a multitude of
restricted domains, in each of which it is clear how to proceed. A general
theory of how the three strategies of backward-driven search, forward-driven
search and combination of methods can be combined successfully to capture common
mathematical inferences is needed. Insofar as structural features of the
mathematics that we care about make it possible to preceed in principled and 
effective ways, we should identify these structural features.

It is also common in mathematics that one first recognizes an algebraic 
structure in a domain of interest, and then one instantiates facts, procedures
and methods that have been developed in the general setting to the case at hand.
Thus, algebraic reasoning involves complementary processes of abstraction and
instantiation. The most commonly cited advantage of the algebraic method is
generality. Sometimes the benefits are terminological and notational: algebraic
abstraction has a way of focussing our efforts by suppressing distracting
information that is irrelevant to the solution of a problem. Finally, the
algebraic method enables us to discover notions that are likely to be fruitful
elsewhere. It provides a uniform way of seeing analogies in otherwise disparate
settings. In other words, they lead us to make investigations, they are the
expression of interest and they direct our interest.

In the example of group theory, once we recognize that we are dealing with a
group, facts about groups are suddenly ready to hand. We know how to simplify
terms, and what properties are potentially relevant. The logical story does
not have much to say about how this works. Nor does it have much to say about
how we are able to reason in the abstract setting, and how this reasoning differs
from that of the domain of application. In formal verification, the notion of
a collection of facts and procedures that are ready at hand is sometimes called
a context.

To enable contexts locally, proof assistants have developed schemes that go
under names such as locales or modules. In these locales or modules, the system
has to provide mechanisms for identifying certain theorems as theorems about
groups, for specializing facts about groups automatically to the specific
group at hand, and for keeping track of the calculational rules and basic facts
that enable the automated tools to recognize straightforward inferences and
calculations in both the general and particular settings. Furthermore, one needs
mechanisms to manage locales and combine them effectively. An amount of
bookkeeping is needed to keep track of the facts and procedures that are
immediately available to human mathematicians, when we recognize one algebraic
structure to be present in another. A good deal more effort is needed to
determine what lies behind even the most straightforward algebraic inferences,
and how we understand algebraic proofs.
\end{document}
