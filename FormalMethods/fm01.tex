\documentclass{article}

\title{Formal Methods homework set \hwnumber}
\author{Egbert Rijke}
\date\today

\usepackage{mathpazo}
\usepackage{amsmath,amsthm,amssymb}
\usepackage{xcolor}
\usepackage{comment}

\newcommand{\note}[1]{{\color{red}#1}}

\newcommand{\sigalg}{\mathcal}

\theoremstyle{definition}
\newtheorem{ex}{Exercise}



\newcommand\hwnumber{1}

\begin{document}

\maketitle

\begin{ex}
%set local notation
\newcommand{\ulimsym}{{\uparrow}}
\newcommand{\ulim}[2]{\ulimsym_{#1\in\mathbb{N}}\,{#2}_{#1}}
\newcommand{\dlimsym}{{\downarrow}}
\newcommand{\dlim}[2]{\dlimsym_{#1\in\mathbb{N}}\,{#2}_{#1}}
Let $X$ be a space, let $\sigalg{A}$ be a $\sigma$-algebra on $X$ and
let $P:\sigalg{A}\to[0,1]$ be a function with $P(\emptyset)=0$ and $P(X)=1$. The exercise asks to verify that the three properties
\begin{enumerate}
\item $P$ is $\sigma$-additive: let $PD_\sigalg{A}:=\{s:\mathbb{N}\to\sigalg{A}\mid i\neq j \Rightarrow s_i\cap s_j=\emptyset\text{ for all }i,j\in\mathbb{N}\}$ be the collection of pairwise disjoint sequences of measurable sets; for every $S\in PD_\sigalg{A}$ we have
\begin{equation*}
\textstyle
P(\bigcup_{i\in\mathbb{N}}S_i)=\sum_{i\in\mathbb{N}}P(S_i)
\end{equation*}
\item $P$ is upwards continuous: let $MI_\sigalg{A}:=\in\{s:\mathbb{N}\to\sigalg{A}\mid s_i\subseteq s_{i+1}\text{ for every }i\in\mathbb{N}\}$ be the collection of monotonously increasing sequences of measurable sets; for every $S\in MI_\sigalg{A}$ we have
\begin{equation*}
\textstyle
P(\bigcup_{i\in\mathbb{N}}S_i)=\ulimsym_{i}\,{P(S_i)}
\end{equation*}
\item $P$ is downwards continuous: let $MD_\sigalg{A}:=\{s:\mathbb{N}\to\sigalg{A}\mid s_{i+1}\subseteq s_i\text{ for every }i\in\mathbb{N}\}$ be the collection of monotonously decreasing sequences of measurable sets; for every $S\in MD_\sigalg{A}$ we have
\begin{equation*}
\textstyle
P(\bigcap_{i\in\mathbb{N}}S_i)=\dlim{i}{P(S_i)}
\end{equation*}
\end{enumerate}
are equivalent.

Suppose first that $P$ is $\sigma$-additive and let $S\in MI_\sigalg{A}$. Then we have the sequence $S':\mathbb{N}\to\sigalg{A}$ given by $S'_0:=S_0$ and $S'_{i+1}:=S_{i+1}\setminus S_i$ which is pairwise disjoint because $S$ is increasing. Since $P$ is assumed to be $\sigma$-additive, we have the equality
\begin{equation*}
\textstyle
P(\bigcup_{i\in\mathbb{N}}S_i)=\sum_{i\in\mathbb{N}}P(S_i).
\end{equation*}
By definition, we have the equality
\begin{equation*}
\sum_{i\in\mathbb{N}}P(S_i)=\lim_{j\to\infty}\sum_{i=0}^j P(S_i).
\end{equation*}
Since $P$ is assumed to take values in $[0,1]$ it is always non-negative, so the latter is a limit of a monotone increasing sequence of real numbers, which we may write in this exercise as $\ulimsym_j\,\sum_{i=0}^j P(S_i)$. Now we note that 
\begin{equation*}
\textstyle
\sum_{i=0}^j P(S_i) = P(\bigcup_{i=0}^j S_i) = P(S_j)
\end{equation*}
by the $\sigma$-additivity of $P$ ($\sigma$-additivity implies finite additivity). We conclude that $P(\bigcup_{i\in\mathbb{N}}S_i)=\ulimsym_{j}\,{P(S_j)}$.

Now suppose that $P$ is upwards continuous. We will show that it follows that $P$ is $\sigma$-additive. To begin, let $S\in PD_\sigalg{A}$. Then we construct the monotone increasing sequence $S'$ by $S'_i=\bigcup_{j=0}^i S_j$. Now it follows by the upwards continuity of $P$ that
\begin{equation*}
\textstyle
P(\bigcup_{i\in\mathbb{N}}S'_i)=\ulimsym_{i}\,{P(S'_i)}
\end{equation*}

\note{We probably have to assume finite additivity of $P$.\\ It is not clear what is meant by the phrase `natural approximations from without or within'.}
\end{ex}

\begin{ex}
Answer to second exercise.
\end{ex}

\setcounter{ex}{0}

\begin{ex}
Answer to first exercise in second part.
\end{ex}

\end{document}
