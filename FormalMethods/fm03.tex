\documentclass{article}

\title{Formal Methods homework set \hwnumber}
\author{Egbert Rijke}
\date\today

\usepackage{mathpazo}
\usepackage{amsmath,amsthm,amssymb}
\usepackage{xcolor}
\usepackage{comment}

\newcommand{\note}[1]{{\color{red}#1}}

\newcommand{\sigalg}{\mathcal}

\theoremstyle{definition}
\newtheorem{ex}{Exercise}



\newcommand\hwnumber{3}
\newcommand{\cprob}[2]{\prob{#1\mathbin|#2}}
\newcommand{\prob}[1]{P(#1)}
\newcommand{\indep}[2]{#1\mathbin{\amalg}#2}
\newcommand{\cindep}[3]{\indep{#1}{#2}\mid #3}

\begin{document}

\maketitle

\begin{ex}
\begin{enumerate}
\item The Markov decomposition can be found by listing all the probabilities of
all events given their parents. So we obtain
\begin{equation*}
\cprob{A}{F}
  \cdot
\cprob{B}{A,C}
  \cdot
\prob{C}
  \cdot
\prob{D}
  \cdot
\cprob{E}{B,C,D}
  \cdot
\prob{F}
  \cdot
\cprob{G}{F,D}.
\end{equation*}
\item $2+4+1+1+8+1+4=21$.
\item Note that $\cindep{D}{A}{B,F,G}$ and $\cindep{D}{B}{F,G}$. Therefore,
we deduce that
\begin{equation*}
\cprob{D=1}{A=0,B=1,F=1,G=0}=\cprob{D=1}{F=1,G=0}.
\end{equation*}
Note also that $\indep{D}{F}$, so we can compute
\begin{align*}
\cprob{D=1}{F=1,G=0}
  & =
\frac
  {\cprob{G=0}{F=1,D=1}\cdot\cprob{D=1}{F=1}}
  {\cprob{G=0}{F=1}}
  \\
  & =
\frac
  {\cprob{G=0}{F=1,D=1}\cdot\prob{D=1}}
  {\cprob{G=0}{F=1}}.
\end{align*}
The terms in the numerator are terms of the Markov decomposition, so it remains 
to compute $\cprob{G=0}{F=1}$. 
\begin{align*}
\cprob{G=0}{F=1}
  & =
\sum_d \cprob{G=0\land D=d}{F=1}
  \\
  & =
\sum_d \cprob{G=0}{D=d,F=1}\cprob{D=d}{F=1}
  \\
  & =
\sum_d \cprob{G=0}{D=d,F=1}\prob{D=d}.
\end{align*}
Thus, we express $\cprob{D=1}{A=0,B=1,F=1,G=0}$ in terms of the variables
appearing in the Markov decomposition in the following way:
\begin{equation*}
\frac
  {\cprob{G=0}{F=1,D=1}\cdot\prob{D=1}}
  {\sum_d \cprob{G=0}{D=d,F=1}\prob{D=d}}.
\end{equation*}
\item Every arrow except the arrow from $F$ to $A$ appears in an unshielded
collider, so the Markov equivalence class contains the given graph, and the
given graph with the edge from $F$ to $A$ reversed.
\item The Markov blanket contains the parents, the children and the parents of
the children, so the Markov blanket of $C$ consists of $A$, $B$, $D$ and $E$.
\item In that case, there is no edge from $F$ to $D$ or vice versa, but there
will be an edge from $F$ to $B$. The Markov blanket therefore consists of $B$
and $C$.
\end{enumerate}
\end{ex}

\begin{ex}
\begin{enumerate}
\item There is the active path $B-A-C-G$, so $B$ and $G$ are not d-separated.
\item Yes, because the Markov blanket of $F$ consists exactly of $C$ and $G$.
\item All three of $C$, $D$ and $F$ appear in colliders with $G$ at one of the
ends. So none of these can d-separate $A$ and $G$. Therefore, we find
$\varnothing$, $\{B\}$, $\{E\}$ and $\{B,E\}$. 
\item There is no active path from $B$ to $E$, so we have $\indep{B}{E}$.
Therefore, we have
\begin{equation*}
\cprob{E=0}{B=1}=\prob{E=0}.
\end{equation*}
\item Likewise, $\cprob{A=1}{G=1}=\prob{A=1}$. 
\item Note that $\cindep{B}{C}{A}$, so we have
\begin{align*}
\cprob{B=0}{A=1,C=0}
  & =
\cprob{B=0}{A=1}
  \\
  & =
\frac{\cprob{A=1}{B=0}\cdot\prob{B=0}}{\prob{A=1}}
  \\
  & =
\frac{\cprob{A=1}{B=0}\cdot\prob{B=0}}{\sum_b\cprob{A=1}{B=b}\prob{B=b}}.
\end{align*}
\end{enumerate}
\end{ex}

\begin{ex}
\begin{enumerate}
\item No, there is the active path $X4-X1-X2$.
\item No, there is the active path $X8-X1-X7-X6$.
\item Yes, all the paths from $X8$ to $X7$ go through either 
$X1\rightarrow X6\leftarrow X7$, $X2\rightarrow X3\leftarrow X7$ or
$X2\rightarrow X5\leftarrow X3$, so $X8$ and $X7$ are unconditionally 
independent.
\item The set $\{X1\}$ d-separates $X7$ and $X2$. The only smaller set is
the empty set, but $\varnothing$ does not d-separate $X7$ and $X2$ because
they are d-connected by the path $X7-X1-X2$.
\item $X4-X8$ and $X1-X2$ are the only edges that are not in a collider pair.
Reverting the $X1-X2$ edge would create new d-separated pairs, such as 
$(X8,X2)$. Therefore, the only edge that may be reverted is $X4-X8$.
\item Yes, the path $X8 \rightarrow X1 \rightarrow X2 \rightarrow X5
\leftarrow X3 \leftarrow X7$ is active, because the only collider it goes through
is at $X5$.
\item By intervention on $X1$, the edges into $X1$ are deleted. Thus, by intervening
on $X1$, there remain no paths from $X7$ to $X8$.
\item Yes, via the path $X7-X6-X1$.
\item $\{X1\}$. 
\end{enumerate}
\end{ex}

\begin{ex}
\begin{enumerate}
\item
\begin{equation*}
\begin{tikzcd}
S
  &
Q \ar{l}
  \ar[-]{d}
  \\
U \ar{u}
  \ar[-]{r}
  &
R \ar{r}
  &
V
  &
T \ar{l}
\end{tikzcd}
\end{equation*}
Removing the edges for each of the independencies is obvious. Then $S$ and $V$ 
have to be the only colliders because only $S$ and $V$ do not appear in their
respective conditioning sets.
\item The Markov blanket of $R$ consists of $Q$, $T$, $U$ and $V$. Note that
$S$ is not a parent, and neither it is a child of $R$, so it cannot be in the
Markov blanket.
\end{enumerate}
\end{ex}

\begin{ex}
\begin{enumerate}
\item By one run of removing edges and labeling unshielded colliders, we obtain
the graph
\begin{equation*}
\begin{tikzcd}
X1 
  \ar[-]{d}
  \ar{dr}
  \\
X4 \ar[-]{r}
  &
X3 
  &
X2
  \ar{l}
\end{tikzcd}
\end{equation*}
Second, $X2\to X3 - X4$ cannot be an unshielded collider, so the edge $X3-X4$
must be directed as $X3\to X4$. Then, to prevent the existence of a cycle
and to prevent that $X1$ and $X2$ become d-connected, we must direct the
edge $X1-X4$ as $X1\to X4$. The resulting equivalence class consists only of
the graph
\begin{equation*}
\begin{tikzcd}
X1 
  \ar{d}
  \ar{dr}
  \\
X4 
  &
X3 \ar{l}
  &
X2
  \ar{l}
\end{tikzcd}
\end{equation*}
\end{enumerate}
\end{ex}

\begin{ex}
Since all edges remain, there could not have been arrows pointing into $X1$.
that means we're left with the equivalence class
\begin{equation*}
\begin{tikzcd}[column sep=tiny]
{} & X1 \ar{dl} \ar{dr}
  \\
X2 \ar[-]{rr}
  & &
X3
\end{tikzcd}
\end{equation*}
Note that directing the edge $X2-X3$ in either way cannot change the equivalence
class, since it does neither change the adjacencies nor the unshielded colliders.
\end{ex}

\begin{ex}
The Markov equivalence class is given by
\begin{equation*}
\begin{tikzcd}[column sep=tiny]
X1
  \ar{dr}
  &
  {} &
X2
  \ar{dl}
  \\
  {} &
X3
  \ar{dr}
  \ar{dl}
  \\
X4
  \ar[-]{rr}
  &
  {} &
X5
\end{tikzcd}
\end{equation*}
Note that we have the unshielded collider $X1\rightarrow X3\leftarrow X2$ because
$X1$ and $X2$ are unconditionally independent. Furthermore, neither
$X1\to X3-X4$ nor $X2\to X3-X5$ can be an unshielded collider, so we find that
the edges $X3\to X4$ and $X3\to X5$ are directed as indicated.
\end{ex}

\begin{ex}
\begin{enumerate}
\item $81$. There are three possibilities for each arrow with a circle and an
arrow tip. We do not have to consider that latencies coincide, and neither
do unshielded colliders $L\rightarrow X\leftarrow L'$ between latencies 
influence the independencies.
\item Only the graph
\begin{equation*}
\begin{tikzcd}
X1
  &
X4 \ar{l} \ar{d}
  \\
X3 \ar{r} \ar{u}
  &
X2
\end{tikzcd}
\end{equation*}
is possible.
\end{enumerate}
\end{ex}

\begin{ex}
\begin{enumerate}
\item By the uniform structure prior, we have $P(G)=P(H)$ for any two graphs
$G$ and $H$. For any data input $d$, we have $\cprob{d}{G}=\cprob{d}{H}$.
\end{enumerate}
\end{ex}

\begin{ex}
\begin{enumerate}
\item As we have seen in class, we have
\begin{align*}
\rho_{12}\rho_{34}
  & = 
\alpha_1\alpha_2\alpha_3\alpha_4
  \\
\rho_{13}\rho_{24}
  & =
\alpha_1\alpha_2\alpha_3\alpha_4e^2
  \\
\rho_{14}\rho_{23}
  & =
\alpha_1\alpha_2\alpha_3\alpha_4e^2.
\end{align*}
So the tetrad conditions are that $\rho_{12}\rho_{34}\neq\rho_{13}\rho_{24}$ and
$\rho_{13}\rho_{24}=\rho_{14}\rho_{23}$. 
\item In the one-latent model, we would have that
$\rho_{12}\rho_{34}=\rho_{13}\rho_{24}=\rho_{14}\rho_{23}$.
\item We would have 
\begin{align*}
\rho_{12}\rho_{34}
  & = 
\alpha_1\alpha_2\alpha_3\alpha_4
  \\
\rho_{13}\rho_{24}
  & =
(\alpha_1\alpha_3e+\beta)(\alpha_2e\alpha_4)
  \\
\rho_{14}\rho_{23}
  & =
(\alpha_1e\alpha_4)(\alpha_2 e\alpha_3+\alpha_2\alpha_1\beta).
\end{align*}
None of them are equal.
\item We would have 
\begin{align*}
\rho_{12}\rho_{34}
  & =
(\alpha_1\alpha_2+\beta)(\alpha_3\alpha_4)
  \\
\rho_{14}\rho_{23}
  & =
(\alpha_1e\alpha_4)(\alpha_2e\alpha_3+\beta\alpha_1e\alpha_3)
  \\
\rho_{13}\rho_{24}
  & =
(\alpha_1e\alpha_3)(\alpha_2e\alpha_4+\beta\alpha_1e\alpha_4).
\end{align*}
So we have $\rho_{13}\rho_{24}=\rho_{14}\rho_{23}$, which are both different
from $\rho_{12}\rho_{34}$. Based on the (in)equalities only, we cannot different
this situation from the situation in part (a).
\end{enumerate}
\end{ex}
\end{document}
