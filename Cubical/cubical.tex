\documentclass{article}

\title{Formal Methods homework set \hwnumber}
\author{Egbert Rijke}
\date\today

\usepackage{mathpazo}
\usepackage{amsmath,amsthm,amssymb}
\usepackage{xcolor}
\usepackage{comment}

\newcommand{\note}[1]{{\color{red}#1}}

\newcommand{\sigalg}{\mathcal}

\theoremstyle{definition}
\newtheorem{ex}{Exercise}



\begin{document}

\begin{abstract}
This is a write-up of Steve's ideas about the (cartesian) cubical category.
\end{abstract}

\begin{defn}
We define the \emph{cubical category} $\cub$ by
\begin{itemize}
\item The objects of $\cub$ are $\cubobj{n}$, for $n\in\mathbb{N}$, where $\cubobj{n}$ is the set $\bool^n$. The object $\cubobj{n}$ is called the \emph{$n$-cube} and elements of $\cubobj{n}$ are its \emph{vertices}.
\item The morphisms are those generated by
\begin{itemize}
\item The \emph{face maps} $\cubface{k}{i}:\cubobj{n}\to\cubobj{n+1}$, where $i\in \bool$ and
$1\leq k\leq n+1$, defined by
\begin{equation*}
\cubface{k}{i}(v)_j \defeq
\begin{cases}
v_j & \text{for }1\leq j<k\\
i & \text{for }j=k\\
v_{j-1} & \text{for }k+1\leq j<n+1.
\end{cases}
\end{equation*}
\item The \emph{degeneracy maps} $\cubdeg{k}:\cubobj{n+1}\to\cubobj{n}$, where
$1\leq k\leq n+1$, defined by
\begin{equation*}
\cubdeg{k}(v)_{j} \defeq
\begin{cases}
v_{j} & \text{for }1\leq j<k\\
v_{j+1} & \text{for }k\leq j\leq n.
\end{cases}
\end{equation*}
\item The \emph{symmetry maps} $\cubsym{\alpha}:\cubobj{n}\to\cubobj{n}$, where
$\alpha\in\aut{\finset{n}}$, defined by
\begin{equation*}
\cubsym{\alpha}(v)_{j}\defeq v_{\alpha(j)}.
\end{equation*}
\item The \emph{diagonal maps} $\cubdiag{k}:\cubobj{n}\to\cubobj{n+1}$, where
$1\leq k\leq n$, defined by
\begin{equation*}
\cubdiag{k}(v)_{j} \defeq
\begin{cases}
v_{j} & \text{for }1\leq j\leq k\\
v_{j-1} & \text{for }k+1\leq j\leq n+1.
\end{cases}
\end{equation*}
\end{itemize}
\end{itemize}
Composition in $\cub$ is ordinary composition of functions.
\end{defn}

\begin{thm}
The category $\cub$ has cartesian products, with $\cubobj{n}\times\cubobj{m}\simeq
\cubobj{n+m}$. Consequently $\cubobj{n}\simeq\cubobj{1}^n$ for every $n\in\mathbb{N}$.
\end{thm}

\begin{proof}
First we have to define $\pi_1:\cubobj{n+m}\to\cubobj{n}$ and $\pi_2:\cubobj{n+m}\to\cubobj{m}$.
We take
\begin{align*}
\pi_1 & \defeq \cubdeg{n+1}\circ\cdots\circ\cubdeg{n+m}\\
\pi_2 & \defeq \cubdeg{1}\circ\cdots\circ\cubdeg{n}.
\end{align*}
We have to show that for $n,m,a\in\mathbb{N}$ and any $f:\cubobj{a}\to\cubobj{n}$
and $g:\cubobj{a}\to\cubobj{m}$ there is a unique $(f,g):\cubobj{a}\to\cubobj{n+m}$
such that $\pi_1\circ(f,g)=f$ and $\pi_2\circ(f,g)=g$. Given 
$f:\cubobj{a}\to\cubobj{n}$, we perform the construction of $(f,g)$ by
induction over $g:\cubobj{a}\to\cubobj{m}$. 

When $g=\catid{\cubobj{a}}$ we define
\begin{equation*}
(f,\catid{\cubobj{a}})=\catface{n+a}f.
\end{equation*} 
Now let $g:\cubobj{a}\to\cubobj{m}$ be such that there is a unique
$(f,g):\cubobj{a}\to\cubobj{n+m}$ with the property that...

Then for $\cubface{k}{i}\circ g$ we define
\begin{equation*}
(f,\cubface{k}{i}\circ g)\defeq \cubface{k+n}{i}\circ(f,g) 
\end{equation*}
\end{proof}

\end{document}
