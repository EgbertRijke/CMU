\documentclass[handout]{beamer}

\mode<presentation>

\title{The groupoid interpretation of type theory}
\author[Egbert]{Egbert Rijke \\ \texttt{erijke@andrew.cmu.edu}}
\date{September 4th, 2014}
\institute{Carnegie Mellon University}

\begin{document}

\begin{frame}
  \titlepage
\end{frame}

\begin{frame}
  \frametitle{Introduction}
  This is a presentation of Hofmann and Streicher's classic paper
  \begin{center}
  \emph{The groupoid interpretation of type theory}
  \end{center}
  originally published in
  \emph{Twenty-five years of constructive type theory (Venice, 1995)}
  in 1998.
  \\
  \vspace{\baselineskip}
  In this paper, the question whether Uniqueness of Identity Proofs is derivable
  in Martin-L\"of type theory, is answered in the negative by exhibiting a
  counter model.
  \\
  \vspace{\baselineskip}
  In these slides I will cite from their paper using notation from the
  HoTT book.
\end{frame}

\begin{frame}
  \frametitle{Outline}
  \tableofcontents[pausesections]
\end{frame}

\section{Syntax}

\begin{frame}
  \frametitle{Identity types}
The intensional identity sets are given by the following constants.
\begin{itemize}
\item  
\end{itemize}
\end{frame}

\begin{frame}
\begin{itemize}
\item<1-> (p. 4) The main purpose of propositional equality is that it can be 
  assumed in contexts and thus allows for hypothetical equality reasoning. In 
  particular, propositional equality can be established by induction.
\item<2-> (p. 4) Definitional equality, on the other hand, can only be 
  established by pure equational reasoning, i.e.~corresponds to the equational
  theory generated by the postulated equality judgements.
\item<3-> (p. 4) Accordingly, definitional equality is (at least in traditional
  cases) decidable, whereas propositional equality is not, as soon as one 
  includes natural numbers and $\Pi$-sets.
\item<4-> (p. 4) By the congruence rules for definitional equality, the latter
  always entails the propositional one, but not necessarily vice versa.
\end{itemize}
\end{frame}

\section{Syntactic considerations on identity sets}

\section{The groupoid interpretation}

\section{Applications and extension}

\end{document}
