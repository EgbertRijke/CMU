\documentclass[handout,xcolor=dvipsnames]{beamer}
\usepackage{xspace}

\input{macros}
\renewcommand\type{\mathsf{Type}}
\newcommand\J{\mathsf{J}}
\newcommand\quotepage[1]{\emph{(p.~#1)}~}

\mode<presentation>

\title{The groupoid interpretation of type theory}
\author[Egbert]{Egbert Rijke \\ \texttt{erijke@andrew.cmu.edu}}
\date{September 4th, 2014}
\institute{Carnegie Mellon University}

\begin{document}

\begin{frame}
  \titlepage
\end{frame}

\begin{frame}
  \frametitle{Introduction}
  This is a presentation of Hofmann and Streicher's classic paper
  \begin{center}
  \emph{The groupoid interpretation of type theory}
  \end{center}
  originally published in
  \emph{Twenty-five years of constructive type theory (Venice, 1995)}
  in 1998.
  \\
  \vspace{\baselineskip}
  In this paper, the question whether Uniqueness of Identity Proofs is derivable
  in Martin-L\"of type theory, is answered in the negative by exhibiting a
  counter model.
  \\
  \vspace{\baselineskip}
  In these slides I will cite from their paper using notation from the
  HoTT book. In particular, I will write $\type$ rather than $\set$.
\end{frame}

\begin{frame}
  \frametitle{Outline}
  \tableofcontents[pausesections]
\end{frame}

\section{Syntax}

\begin{frame}
  \frametitle{Identity types}
\quotepage{4} The intensional identity sets are given by the following constants.
\begin{align*}
\idtypevar{} 
& :\prd{A:\type}{a_1,a_2:A}\type
  \\
\pause
\refl{}
& :\prd{A:\type}{a:A}\idtype{A}{a}{a}
  \\
\pause
\J 
& : \prd{A:\type}{C:\prd{a_1,a_2:A}{s:\idtype{A}{a_1}{a_2}}\type}{d:\prd{a:A}C(a,a,\refl{a})}
  \\ 
& {}\hspace{4em} \prd{a_1,a_2:A}{s:\idtype{A}{a_1}{a_2}}C(a_1,a_2,s).
\end{align*}
\pause
where we impose the definitional equality
\begin{equation*}
\J(d,a,a,\refl{a})\jdeq d(a) : C(a,a,\refl{a})
\end{equation*}
for $A$, $C$, $d$, $a$ of appropriate type.
\end{frame}

\begin{frame}
\begin{itemize}
\item<1-> \quotepage{4} The main purpose of propositional equality is that it can be 
  assumed in contexts and thus allows for hypothetical equality reasoning. In 
  particular, propositional equality can be established by induction.
\item<2-> \quotepage{4} Definitional equality, on the other hand, can only be 
  established by pure equational reasoning, i.e.~corresponds to the equational
  theory generated by the postulated equality judgements.
\item<3-> \quotepage{4} Accordingly, definitional equality is (at least in traditional
  cases) decidable, whereas propositional equality is not, as soon as one 
  includes natural numbers and $\Pi$-sets.
\item<4-> \quotepage{4} By the congruence rules for definitional equality, the latter
  always entails the propositional one, but not necessarily vice versa.
\end{itemize}
\end{frame}

\section{Syntactic considerations on identity sets}

\begin{frame}
  \frametitle{Propositional equality as isomorphism}
\begin{itemize}
\item<1-> It has been argued in (Hofmann 1995) that intensional type theory
augmented by $UIP$ together with an extensionality axiom for $\Pi$-sets
can simulate extensional type theory whilst retaining decidability of type
checking.
\item<2-> On the other hand, we will demonstrate below that Martin-L\"of's
original formulation of identity sets allows for the addition of axioms
(inconsistent with $UIP$) expressing a view of propositional equality as a
generalised notion of isomorphism. Intuitively, these axioms state that for a
universe $U$ and $A,B:U$ the identity set $\idtype{U,A,B}$ corresponds to the
set of isomorphisms between $A$ and $B$. Such version of identity sets may be
useful for a formulation of category theory inside type theory providing a
formal underpinning for the common practice of considering isomorphic objects
as equal.
\end{itemize}
\end{frame}

\section{The groupoid interpretation}

\begin{frame}
  \frametitle{Groupoids}
\begin{definition}
A \emph{groupoid} is a category where all morphisms are isomorphisms. The
groupoids together with functors between them form a (large) category $GPD$. 
\end{definition}

\pause
\begin{definition}
A \emph{family of groupoids} indexed over a groupoid $\Gamma$ is a functor
$A:\Gamma\to GPD$. We write $Ty(\Gamma)$ for the collection of families of
groupoids over $\Gamma$. 
\end{definition}

\pause
When $f:\Delta\to\Gamma$ is a morphism in $GPD$ and $A\in Ty(\Gamma)$,
then the composition $A\circ f$ is an element of $Ty(\Delta)$. We use the
notation $A\{f\}$ for this family.
\end{frame}

\begin{frame}
\begin{example}
For each set $X$ there is a discrete groupoid $\triangle(X)$ with only identity
morphisms.

If $\Gamma$ is a groupoid then a family $I_\Gamma$ indexed over $\Gamma\times
\Gamma$ is defined by $I_\Gamma(\gamma_1,\gamma_2):=\triangle(\Gamma(\gamma_1,
\gamma_2))$ and $I_\Gamma(p_1,p_2)(q):= p_2\circ q \circ p_1^{-1}$, where
$p_i:\gamma_i\to\gamma_i'$ and $q:\gamma_1\to\gamma_2$.
\end{example}

\begin{definition}
Let $A\in Ty(\Gamma)$ be a family of groupoids over $\Gamma$. A
\emph{(dependent) object $M$ of $A$} consists of the following data
\begin{itemize}
\item an object $M(\gamma)$ of $A(\gamma)$ for each $\gamma\in\Gamma$.
\item a morphism $M(p):A(p,M(\gamma))\to M(\gamma')$ in $A(\gamma')$ for each
morphism $p:\gamma\to\gamma'$
\end{itemize}
such that $M(id_\gamma)=id_{M(\gamma)}$ and $M(p'\circ p)=M(p')\circ A(p',M(p))$.

We will write $Tm(A)$ for the collection of dependent objects of $A$. For a
functor $f:\Delta\to\Gamma$ the operation $\_\{f\}:Ty(\Gamma)\to Ty(\Delta)$
extends to dependent objects. If $A\in Ty(\Gamma)$ and $a\in Tm(A)$, then
$a\{f\}\in Ty(A\{f\})$ is given by composing the components of $a$ with $f$ in
the obvious way.
\end{definition}
\end{frame}

\section{Applications and extension}

\end{document}
