\documentclass{beamer}

\mode<presentation>

\title{The groupoid interpretation of type theory}
\author[Egbert]{Egbert Rijke \\ \texttt{erijke@andrew.cmu.edu}}
\date{September 4th, 2014}
\institute{Carnegie Mellon University}

\begin{document}

\begin{frame}
  \titlepage
\end{frame}

\begin{frame}
  \frametitle{Introduction}
  This is a presentation of Hofmann and Streicher's classic paper
  \begin{center}
  \emph{The groupoid interpretation of type theory}
  \end{center}
  originally published in
  \emph{Twenty-five years of constructive type theory (Venice, 1995)}
  in 1998.
  \\
  \vspace{\baselineskip}
  In this paper, the question whether Uniqueness of Identity Proofs is derivable
  in Martin-L\"of type theory, is answered in the negative by exhibiting a
  counter model.
  \\
  \vspace{\baselineskip}
  In these slides I will cite from their paper using notation from the
  HoTT book.
\end{frame}

\begin{frame}
  \frametitle{Outline}
  \tableofcontents[pausesections]
\end{frame}

\section{Syntactic considerations on identity sets}

\section{The groupoid interpretation}

\section{Applications and extension}

\end{document}
